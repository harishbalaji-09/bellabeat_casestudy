% Options for packages loaded elsewhere
\PassOptionsToPackage{unicode}{hyperref}
\PassOptionsToPackage{hyphens}{url}
%
\documentclass[
]{article}
\usepackage{amsmath,amssymb}
\usepackage{iftex}
\ifPDFTeX
  \usepackage[T1]{fontenc}
  \usepackage[utf8]{inputenc}
  \usepackage{textcomp} % provide euro and other symbols
\else % if luatex or xetex
  \usepackage{unicode-math} % this also loads fontspec
  \defaultfontfeatures{Scale=MatchLowercase}
  \defaultfontfeatures[\rmfamily]{Ligatures=TeX,Scale=1}
\fi
\usepackage{lmodern}
\ifPDFTeX\else
  % xetex/luatex font selection
\fi
% Use upquote if available, for straight quotes in verbatim environments
\IfFileExists{upquote.sty}{\usepackage{upquote}}{}
\IfFileExists{microtype.sty}{% use microtype if available
  \usepackage[]{microtype}
  \UseMicrotypeSet[protrusion]{basicmath} % disable protrusion for tt fonts
}{}
\makeatletter
\@ifundefined{KOMAClassName}{% if non-KOMA class
  \IfFileExists{parskip.sty}{%
    \usepackage{parskip}
  }{% else
    \setlength{\parindent}{0pt}
    \setlength{\parskip}{6pt plus 2pt minus 1pt}}
}{% if KOMA class
  \KOMAoptions{parskip=half}}
\makeatother
\usepackage{xcolor}
\usepackage[margin=1in]{geometry}
\usepackage{color}
\usepackage{fancyvrb}
\newcommand{\VerbBar}{|}
\newcommand{\VERB}{\Verb[commandchars=\\\{\}]}
\DefineVerbatimEnvironment{Highlighting}{Verbatim}{commandchars=\\\{\}}
% Add ',fontsize=\small' for more characters per line
\usepackage{framed}
\definecolor{shadecolor}{RGB}{248,248,248}
\newenvironment{Shaded}{\begin{snugshade}}{\end{snugshade}}
\newcommand{\AlertTok}[1]{\textcolor[rgb]{0.94,0.16,0.16}{#1}}
\newcommand{\AnnotationTok}[1]{\textcolor[rgb]{0.56,0.35,0.01}{\textbf{\textit{#1}}}}
\newcommand{\AttributeTok}[1]{\textcolor[rgb]{0.13,0.29,0.53}{#1}}
\newcommand{\BaseNTok}[1]{\textcolor[rgb]{0.00,0.00,0.81}{#1}}
\newcommand{\BuiltInTok}[1]{#1}
\newcommand{\CharTok}[1]{\textcolor[rgb]{0.31,0.60,0.02}{#1}}
\newcommand{\CommentTok}[1]{\textcolor[rgb]{0.56,0.35,0.01}{\textit{#1}}}
\newcommand{\CommentVarTok}[1]{\textcolor[rgb]{0.56,0.35,0.01}{\textbf{\textit{#1}}}}
\newcommand{\ConstantTok}[1]{\textcolor[rgb]{0.56,0.35,0.01}{#1}}
\newcommand{\ControlFlowTok}[1]{\textcolor[rgb]{0.13,0.29,0.53}{\textbf{#1}}}
\newcommand{\DataTypeTok}[1]{\textcolor[rgb]{0.13,0.29,0.53}{#1}}
\newcommand{\DecValTok}[1]{\textcolor[rgb]{0.00,0.00,0.81}{#1}}
\newcommand{\DocumentationTok}[1]{\textcolor[rgb]{0.56,0.35,0.01}{\textbf{\textit{#1}}}}
\newcommand{\ErrorTok}[1]{\textcolor[rgb]{0.64,0.00,0.00}{\textbf{#1}}}
\newcommand{\ExtensionTok}[1]{#1}
\newcommand{\FloatTok}[1]{\textcolor[rgb]{0.00,0.00,0.81}{#1}}
\newcommand{\FunctionTok}[1]{\textcolor[rgb]{0.13,0.29,0.53}{\textbf{#1}}}
\newcommand{\ImportTok}[1]{#1}
\newcommand{\InformationTok}[1]{\textcolor[rgb]{0.56,0.35,0.01}{\textbf{\textit{#1}}}}
\newcommand{\KeywordTok}[1]{\textcolor[rgb]{0.13,0.29,0.53}{\textbf{#1}}}
\newcommand{\NormalTok}[1]{#1}
\newcommand{\OperatorTok}[1]{\textcolor[rgb]{0.81,0.36,0.00}{\textbf{#1}}}
\newcommand{\OtherTok}[1]{\textcolor[rgb]{0.56,0.35,0.01}{#1}}
\newcommand{\PreprocessorTok}[1]{\textcolor[rgb]{0.56,0.35,0.01}{\textit{#1}}}
\newcommand{\RegionMarkerTok}[1]{#1}
\newcommand{\SpecialCharTok}[1]{\textcolor[rgb]{0.81,0.36,0.00}{\textbf{#1}}}
\newcommand{\SpecialStringTok}[1]{\textcolor[rgb]{0.31,0.60,0.02}{#1}}
\newcommand{\StringTok}[1]{\textcolor[rgb]{0.31,0.60,0.02}{#1}}
\newcommand{\VariableTok}[1]{\textcolor[rgb]{0.00,0.00,0.00}{#1}}
\newcommand{\VerbatimStringTok}[1]{\textcolor[rgb]{0.31,0.60,0.02}{#1}}
\newcommand{\WarningTok}[1]{\textcolor[rgb]{0.56,0.35,0.01}{\textbf{\textit{#1}}}}
\usepackage{graphicx}
\makeatletter
\def\maxwidth{\ifdim\Gin@nat@width>\linewidth\linewidth\else\Gin@nat@width\fi}
\def\maxheight{\ifdim\Gin@nat@height>\textheight\textheight\else\Gin@nat@height\fi}
\makeatother
% Scale images if necessary, so that they will not overflow the page
% margins by default, and it is still possible to overwrite the defaults
% using explicit options in \includegraphics[width, height, ...]{}
\setkeys{Gin}{width=\maxwidth,height=\maxheight,keepaspectratio}
% Set default figure placement to htbp
\makeatletter
\def\fps@figure{htbp}
\makeatother
\setlength{\emergencystretch}{3em} % prevent overfull lines
\providecommand{\tightlist}{%
  \setlength{\itemsep}{0pt}\setlength{\parskip}{0pt}}
\setcounter{secnumdepth}{-\maxdimen} % remove section numbering
\ifLuaTeX
  \usepackage{selnolig}  % disable illegal ligatures
\fi
\IfFileExists{bookmark.sty}{\usepackage{bookmark}}{\usepackage{hyperref}}
\IfFileExists{xurl.sty}{\usepackage{xurl}}{} % add URL line breaks if available
\urlstyle{same}
\hypersetup{
  pdftitle={BELLABEAT CASE STUDY},
  pdfauthor={Harish Balaji K},
  hidelinks,
  pdfcreator={LaTeX via pandoc}}

\title{BELLABEAT CASE STUDY}
\author{Harish Balaji K}
\date{}

\begin{document}
\maketitle

{
\setcounter{tocdepth}{2}
\tableofcontents
}
\hypertarget{summary}{%
\subsection{SUMMARY}\label{summary}}

Bellabeat is a high-tech company that manufactures health-focused smart
products.They offer different smart devices that collect data on
activity, sleep, stress, and reproductive health to empower women with
knowledge about their own health and habits.

\hypertarget{mission-statement}{%
\subsection{MISSION STATEMENT}\label{mission-statement}}

To analyze smart device fitness data to gain insight into how consumers
are using their smart devices and use these insights to guide
Bellabeat's marketing strategy for growth in the global smart device
market

\hypertarget{phase1ask}{%
\subsection{PHASE1:ASK}\label{phase1ask}}

\hypertarget{business-task}{%
\subsubsection{BUSINESS TASK}\label{business-task}}

The primary business goal is to utilize external data on smart water
bottle usage to refine Bellabeat's product development and marketing
strategies within the smart water bottle niche. This effort is focused
on gaining in-depth insights into consumer behaviors and preferences
specific to smart water bottles. By achieving this objective, Bellabeat
aims to optimize its approach, cater effectively to potential smart
water bottle customers, and strategically position its products for
success in the ever-evolving smart water bottle market.

\hypertarget{product}{%
\subsubsection{PRODUCT}\label{product}}

Spring - water bottle \href{URL}{https://bellabeat.com/product/spring/}

\hypertarget{current-trends-of-smart-water-bottles}{%
\subsubsection{CURRENT TRENDS OF SMART WATER
BOTTLES}\label{current-trends-of-smart-water-bottles}}

These trends highlight the evolving landscape of smart water bottle
usage, where personalization, motivation, and technological advancements
play pivotal roles in enhancing the user experience and promoting better
hydration habits. * Easy Customization Modern smart water bottles offer
easy customization of hydration targets based on individual factors like
age, weight, activity levels, and environmental conditions. Users can
set their own hydration preferences and reminders for a personalized
experience. * Motivational and Fun Features These bottles incorporate
motivational elements such as visual cues (e.g., illumination, color
changes) to encourage users to drink water regularly. Social integration
features allow users to engage with friends on social media, fostering
friendly challenges and adding excitement to hydration routines. * Long
Battery Life Many smart water bottles feature long-lasting batteries,
making them suitable for daily use, travel, and outdoor activities.
While smaller models typically have battery capacities of 200-500 mAh
for portability, larger and advanced bottles can reach up to 1500mAh,
providing weeks or months of usage without recharging. * Backed by
Science Clinical trials support the efficacy of smart water bottles in
promoting healthy hydration habits. Beyond reminders, these bottles hold
potential in the medical field, addressing the challenge of maintaining
proper hydration and contributing to overall well-being.

\hypertarget{application-to-bellabeat-customers}{%
\subsubsection{APPLICATION TO BELLABEAT
CUSTOMERS}\label{application-to-bellabeat-customers}}

Spring's app and smart technology can calculate the optimal amount of
water for user's body and remind users of water intake base on the users
age, height, weight, local weather, activity level, pregnancy or
breastfeeding, help to remind users of water consumption. Thus, Spring,
as smart bottle that can help remind users avoid dehydration, establish,
and maintain healthy hydration habit, is considerable product for
development in the market.

\hypertarget{key-stakeholders}{%
\subsubsection{KEY STAKEHOLDERS}\label{key-stakeholders}}

The main stakeholders here are Urška Sršen, Bellabeat's co-founder and
Chief Creative Officer; Sando Mur, Mathematician and Bellabeat's
cofounder; And the rest of the Bellabeat marketing analytics team.

\hypertarget{phase2prepare}{%
\subsection{PHASE2:PREPARE}\label{phase2prepare}}

\hypertarget{data-used}{%
\subsubsection{DATA USED}\label{data-used}}

The data source used for our case study is FitBit Fitness Tracker Data.
This dataset is stored in Kaggle and was made available through Mobius.

\hypertarget{accessibility-and-privacy-of-data}{%
\subsubsection{ACCESSIBILITY AND PRIVACY OF
DATA}\label{accessibility-and-privacy-of-data}}

Verifying the metadata of our dataset we can confirm it is open-source.
The owner has dedicated the work to the public domain by waiving all of
his or her rights to the work worldwide under copyright law, including
all related and neighboring rights, to the extent allowed by law. You
can copy, modify, distribute and perform the work, even for commercial
purposes, all without asking permission.

\hypertarget{information-on-dataset}{%
\subsubsection{INFORMATION ON DATASET}\label{information-on-dataset}}

These datasets were generated by respondents to a distributed survey via
Amazon Mechanical Turk between 03.12.2016-05.12.2016. Thirty eligible
Fitbit users consented to the submission of personal tracker data,
including minute-level output for physical activity, heart rate, and
sleep monitoring. Variation between output represents use of different
types of Fitbit trackers and individual tracking behaviors /
preferences.

\hypertarget{roccc-analysis}{%
\subsubsection{ROCCC ANALYSIS}\label{roccc-analysis}}

\begin{itemize}
\tightlist
\item
  Reliability : LOW -- dataset was collected from 30 individuals whose
  gender is unknown.
\item
  Originality : LOW -- third party data collect using Amazon Mechanical
  Turk.
\item
  Comprehensive : MEDIUM -- dataset contains multiple fields on daily
  activity intensity, calories used, daily steps taken, daily sleep time
  and weight record.
\item
  Current : MEDIUM -- data is 5 years old but the habit of how people
  live does not change over a few years
\item
  Cited : HIGH -- data collector and source is well documented
\end{itemize}

\hypertarget{data-organisation}{%
\subsubsection{DATA ORGANISATION}\label{data-organisation}}

Available to us are 6 CSV documents and a excel sheet created by
research on other smart water bottle products. Each document represents
different quantitative data tracked by Fitbit. The data is considered
long since each row is one time point per subject, so each subject will
have data in multiple rows.Every user has a unique ID and different rows
since data is tracked by day and time. Counted sample size (users) of
each table and verified time length of analysis - 31 days.

\begin{Shaded}
\begin{Highlighting}[]
\NormalTok{data }\OtherTok{\textless{}{-}} \FunctionTok{data.frame}\NormalTok{(}
  \AttributeTok{Filename =} \FunctionTok{c}\NormalTok{(}\StringTok{"dailyActivity\_merged.csv"}\NormalTok{, }\StringTok{"sleepDay\_merged.csv"}\NormalTok{, }\StringTok{"dailySteps\_merged.csv"}\NormalTok{,}\StringTok{"dailyIntensties\_merged.csv"}\NormalTok{, }\StringTok{"dailyCalories\_merged.csv"}\NormalTok{, }\StringTok{"weightLogInfo\_merged.csv"}\NormalTok{,}\StringTok{"Smart\_waterbottles.xlsx"}\NormalTok{),}
  \AttributeTok{TypeOfFile =} \FunctionTok{c}\NormalTok{(}\StringTok{"CSV"}\NormalTok{, }\StringTok{"CSV"}\NormalTok{, }\StringTok{"CSV"}\NormalTok{, }\StringTok{"CSV"}\NormalTok{, }\StringTok{"CSV"}\NormalTok{, }\StringTok{"CSV"}\NormalTok{,}\StringTok{"XLSX"}\NormalTok{),}
  \AttributeTok{Description =} \FunctionTok{c}\NormalTok{(}
    \StringTok{"Daily Activity over 31 days of 33 users. Tracking daily: Steps, Distance, Intensities, Calories"}\NormalTok{,}
    \StringTok{"Daily sleep logs, tracked by: Total count of sleeps a day, Total minutes, Total Time in Bed"}\NormalTok{,}
    \StringTok{"Daily Steps over 31 days of 33 users"}\NormalTok{,}
    \StringTok{"Daily Intensity over 31 days of 33 users. Measured in Minutes and Distance, dividing groups in 4 categories: Sedentary, Lightly Active, Fairly Active,Very Active"}\NormalTok{,}
    \StringTok{"Daily Calories over 31 days of 33 users"}\NormalTok{,}
    \StringTok{"Weight track by day in Kg and Pounds over 30 days. Calculation of BMI.5 users report weight manually 3 users not.In total there are 8 users"}\NormalTok{,}
    \StringTok{"Data on 6 other smart watter bottle products"}
\NormalTok{  )}
\NormalTok{)}
\FunctionTok{print}\NormalTok{(data)}
\end{Highlighting}
\end{Shaded}

\begin{verbatim}
##                     Filename TypeOfFile
## 1   dailyActivity_merged.csv        CSV
## 2        sleepDay_merged.csv        CSV
## 3      dailySteps_merged.csv        CSV
## 4 dailyIntensties_merged.csv        CSV
## 5   dailyCalories_merged.csv        CSV
## 6   weightLogInfo_merged.csv        CSV
## 7    Smart_waterbottles.xlsx       XLSX
##                                                                                                                                                         Description
## 1                                                                   Daily Activity over 31 days of 33 users. Tracking daily: Steps, Distance, Intensities, Calories
## 2                                                                       Daily sleep logs, tracked by: Total count of sleeps a day, Total minutes, Total Time in Bed
## 3                                                                                                                              Daily Steps over 31 days of 33 users
## 4 Daily Intensity over 31 days of 33 users. Measured in Minutes and Distance, dividing groups in 4 categories: Sedentary, Lightly Active, Fairly Active,Very Active
## 5                                                                                                                           Daily Calories over 31 days of 33 users
## 6                       Weight track by day in Kg and Pounds over 30 days. Calculation of BMI.5 users report weight manually 3 users not.In total there are 8 users
## 7                                                                                                                      Data on 6 other smart watter bottle products
\end{verbatim}

\hypertarget{data-credibility-and-integrity}{%
\subsubsection{DATA CREDIBILITY AND
INTEGRITY}\label{data-credibility-and-integrity}}

Due to the limitation of size (30 users) and not having any demographic
information we could encounter a sampling bias. We are not sure if the
sample is representative of the population as a whole. Another problem
we would encounter is that the dataset is not current and also the time
limitation of the survey (2 months long). That is why we will give our
case study an operational approach.

\hypertarget{phase3process}{%
\subsection{PHASE3:PROCESS}\label{phase3process}}

The entire analysis is done in RStudio.

\hypertarget{installing-packages-and-libraries}{%
\subsubsection{INSTALLING PACKAGES AND
LIBRARIES}\label{installing-packages-and-libraries}}

We will choose the packages that will help us on our analysis and open
them. We will use the following packages for our analysis:

\begin{itemize}
\tightlist
\item
  tidyverse
\item
  here
\item
  skimr
\item
  janitor
\item
  lubridate
\item
  ggpubr
\item
  ggrepel
\item
  hms
\end{itemize}

\begin{Shaded}
\begin{Highlighting}[]
\FunctionTok{library}\NormalTok{(ggpubr)}
\end{Highlighting}
\end{Shaded}

\begin{verbatim}
## Loading required package: ggplot2
\end{verbatim}

\begin{Shaded}
\begin{Highlighting}[]
\FunctionTok{library}\NormalTok{(tidyverse)}
\end{Highlighting}
\end{Shaded}

\begin{verbatim}
## -- Attaching core tidyverse packages ------------------------ tidyverse 2.0.0 --
## v dplyr     1.1.3     v readr     2.1.4
## v forcats   1.0.0     v stringr   1.5.0
## v lubridate 1.9.3     v tibble    3.2.1
## v purrr     1.0.2     v tidyr     1.3.0
\end{verbatim}

\begin{verbatim}
## -- Conflicts ------------------------------------------ tidyverse_conflicts() --
## x dplyr::filter() masks stats::filter()
## x dplyr::lag()    masks stats::lag()
## i Use the conflicted package (<http://conflicted.r-lib.org/>) to force all conflicts to become errors
\end{verbatim}

\begin{Shaded}
\begin{Highlighting}[]
\FunctionTok{library}\NormalTok{(here)}
\end{Highlighting}
\end{Shaded}

\begin{verbatim}
## here() starts at C:/Users/Harish/OneDrive/Documents/Rcase_studies/bellabeat_casestudy
\end{verbatim}

\begin{Shaded}
\begin{Highlighting}[]
\FunctionTok{library}\NormalTok{(skimr)}
\FunctionTok{library}\NormalTok{(janitor)}
\end{Highlighting}
\end{Shaded}

\begin{verbatim}
## 
## Attaching package: 'janitor'
## 
## The following objects are masked from 'package:stats':
## 
##     chisq.test, fisher.test
\end{verbatim}

\begin{Shaded}
\begin{Highlighting}[]
\FunctionTok{library}\NormalTok{(lubridate)}
\FunctionTok{library}\NormalTok{(ggrepel)}
\FunctionTok{library}\NormalTok{(readxl)}
\end{Highlighting}
\end{Shaded}

\hypertarget{importing-datasets}{%
\subsubsection{IMPORTING DATASETS}\label{importing-datasets}}

Knowing the datasets we have, we will upload the datasets that will help
us answer our business task. On our analysis we will focus on the
following datasets

\begin{itemize}
\tightlist
\item
  Daily\_activity
\item
  Daily\_intensities
\item
  Daily\_calories
\end{itemize}

Due to the the small sample we won't consider Weight (8 Users) for this
analysis.

\begin{Shaded}
\begin{Highlighting}[]
\NormalTok{daily\_steps }\OtherTok{\textless{}{-}} \FunctionTok{read.csv}\NormalTok{(}\StringTok{"dailySteps\_merged.csv"}\NormalTok{)}
\NormalTok{daily\_intensities }\OtherTok{\textless{}{-}} \FunctionTok{read.csv}\NormalTok{(}\StringTok{"dailyIntensities\_merged.csv"}\NormalTok{)}
\NormalTok{daily\_calories }\OtherTok{\textless{}{-}} \FunctionTok{read.csv}\NormalTok{(}\StringTok{"dailyCalories\_merged.csv"}\NormalTok{)}
\NormalTok{hourly\_steps }\OtherTok{\textless{}{-}} \FunctionTok{read.csv}\NormalTok{(}\StringTok{"hourlySteps\_merged.csv"}\NormalTok{)}
\NormalTok{daily\_activity }\OtherTok{\textless{}{-}} \FunctionTok{read.csv}\NormalTok{(}\StringTok{"dailyActivity\_merged.csv"}\NormalTok{)}
\NormalTok{products }\OtherTok{\textless{}{-}} \FunctionTok{read\_excel}\NormalTok{(}\StringTok{"smart\_waterbottles.xlsx"}\NormalTok{)}
\end{Highlighting}
\end{Shaded}

\hypertarget{preview-dataset}{%
\subsubsection{PREVIEW DATASET}\label{preview-dataset}}

We will preview our selected data frames and check the summary of each
column.

\begin{Shaded}
\begin{Highlighting}[]
\FunctionTok{head}\NormalTok{(daily\_steps)}
\end{Highlighting}
\end{Shaded}

\begin{verbatim}
##           Id ActivityDay StepTotal
## 1 1503960366   4/12/2016     13162
## 2 1503960366   4/13/2016     10735
## 3 1503960366   4/14/2016     10460
## 4 1503960366   4/15/2016      9762
## 5 1503960366   4/16/2016     12669
## 6 1503960366   4/17/2016      9705
\end{verbatim}

\begin{Shaded}
\begin{Highlighting}[]
\FunctionTok{str}\NormalTok{(daily\_steps)}
\end{Highlighting}
\end{Shaded}

\begin{verbatim}
## 'data.frame':    940 obs. of  3 variables:
##  $ Id         : num  1.5e+09 1.5e+09 1.5e+09 1.5e+09 1.5e+09 ...
##  $ ActivityDay: chr  "4/12/2016" "4/13/2016" "4/14/2016" "4/15/2016" ...
##  $ StepTotal  : int  13162 10735 10460 9762 12669 9705 13019 15506 10544 9819 ...
\end{verbatim}

\begin{Shaded}
\begin{Highlighting}[]
\FunctionTok{head}\NormalTok{(daily\_intensities)}
\end{Highlighting}
\end{Shaded}

\begin{verbatim}
##           Id ActivityDay SedentaryMinutes LightlyActiveMinutes
## 1 1503960366   4/12/2016              728                  328
## 2 1503960366   4/13/2016              776                  217
## 3 1503960366   4/14/2016             1218                  181
## 4 1503960366   4/15/2016              726                  209
## 5 1503960366   4/16/2016              773                  221
## 6 1503960366   4/17/2016              539                  164
##   FairlyActiveMinutes VeryActiveMinutes SedentaryActiveDistance
## 1                  13                25                       0
## 2                  19                21                       0
## 3                  11                30                       0
## 4                  34                29                       0
## 5                  10                36                       0
## 6                  20                38                       0
##   LightActiveDistance ModeratelyActiveDistance VeryActiveDistance
## 1                6.06                     0.55               1.88
## 2                4.71                     0.69               1.57
## 3                3.91                     0.40               2.44
## 4                2.83                     1.26               2.14
## 5                5.04                     0.41               2.71
## 6                2.51                     0.78               3.19
\end{verbatim}

\begin{Shaded}
\begin{Highlighting}[]
\FunctionTok{str}\NormalTok{(daily\_intensities)}
\end{Highlighting}
\end{Shaded}

\begin{verbatim}
## 'data.frame':    940 obs. of  10 variables:
##  $ Id                      : num  1.5e+09 1.5e+09 1.5e+09 1.5e+09 1.5e+09 ...
##  $ ActivityDay             : chr  "4/12/2016" "4/13/2016" "4/14/2016" "4/15/2016" ...
##  $ SedentaryMinutes        : int  728 776 1218 726 773 539 1149 775 818 838 ...
##  $ LightlyActiveMinutes    : int  328 217 181 209 221 164 233 264 205 211 ...
##  $ FairlyActiveMinutes     : int  13 19 11 34 10 20 16 31 12 8 ...
##  $ VeryActiveMinutes       : int  25 21 30 29 36 38 42 50 28 19 ...
##  $ SedentaryActiveDistance : num  0 0 0 0 0 0 0 0 0 0 ...
##  $ LightActiveDistance     : num  6.06 4.71 3.91 2.83 5.04 ...
##  $ ModeratelyActiveDistance: num  0.55 0.69 0.4 1.26 0.41 ...
##  $ VeryActiveDistance      : num  1.88 1.57 2.44 2.14 2.71 ...
\end{verbatim}

\begin{Shaded}
\begin{Highlighting}[]
\FunctionTok{head}\NormalTok{(daily\_calories)}
\end{Highlighting}
\end{Shaded}

\begin{verbatim}
##           Id ActivityDay Calories
## 1 1503960366   4/12/2016     1985
## 2 1503960366   4/13/2016     1797
## 3 1503960366   4/14/2016     1776
## 4 1503960366   4/15/2016     1745
## 5 1503960366   4/16/2016     1863
## 6 1503960366   4/17/2016     1728
\end{verbatim}

\begin{Shaded}
\begin{Highlighting}[]
\FunctionTok{str}\NormalTok{(daily\_calories)}
\end{Highlighting}
\end{Shaded}

\begin{verbatim}
## 'data.frame':    940 obs. of  3 variables:
##  $ Id         : num  1.5e+09 1.5e+09 1.5e+09 1.5e+09 1.5e+09 ...
##  $ ActivityDay: chr  "4/12/2016" "4/13/2016" "4/14/2016" "4/15/2016" ...
##  $ Calories   : int  1985 1797 1776 1745 1863 1728 1921 2035 1786 1775 ...
\end{verbatim}

\begin{Shaded}
\begin{Highlighting}[]
\FunctionTok{head}\NormalTok{(hourly\_steps)}
\end{Highlighting}
\end{Shaded}

\begin{verbatim}
##           Id          ActivityHour StepTotal
## 1 1503960366 4/12/2016 12:00:00 AM       373
## 2 1503960366  4/12/2016 1:00:00 AM       160
## 3 1503960366  4/12/2016 2:00:00 AM       151
## 4 1503960366  4/12/2016 3:00:00 AM         0
## 5 1503960366  4/12/2016 4:00:00 AM         0
## 6 1503960366  4/12/2016 5:00:00 AM         0
\end{verbatim}

\begin{Shaded}
\begin{Highlighting}[]
\FunctionTok{str}\NormalTok{(hourly\_steps)}
\end{Highlighting}
\end{Shaded}

\begin{verbatim}
## 'data.frame':    22099 obs. of  3 variables:
##  $ Id          : num  1.5e+09 1.5e+09 1.5e+09 1.5e+09 1.5e+09 ...
##  $ ActivityHour: chr  "4/12/2016 12:00:00 AM" "4/12/2016 1:00:00 AM" "4/12/2016 2:00:00 AM" "4/12/2016 3:00:00 AM" ...
##  $ StepTotal   : int  373 160 151 0 0 0 0 0 250 1864 ...
\end{verbatim}

\begin{Shaded}
\begin{Highlighting}[]
\FunctionTok{head}\NormalTok{(daily\_activity)}
\end{Highlighting}
\end{Shaded}

\begin{verbatim}
##           Id ActivityDate TotalSteps TotalDistance TrackerDistance
## 1 1503960366    4/12/2016      13162          8.50            8.50
## 2 1503960366    4/13/2016      10735          6.97            6.97
## 3 1503960366    4/14/2016      10460          6.74            6.74
## 4 1503960366    4/15/2016       9762          6.28            6.28
## 5 1503960366    4/16/2016      12669          8.16            8.16
## 6 1503960366    4/17/2016       9705          6.48            6.48
##   LoggedActivitiesDistance VeryActiveDistance ModeratelyActiveDistance
## 1                        0               1.88                     0.55
## 2                        0               1.57                     0.69
## 3                        0               2.44                     0.40
## 4                        0               2.14                     1.26
## 5                        0               2.71                     0.41
## 6                        0               3.19                     0.78
##   LightActiveDistance SedentaryActiveDistance VeryActiveMinutes
## 1                6.06                       0                25
## 2                4.71                       0                21
## 3                3.91                       0                30
## 4                2.83                       0                29
## 5                5.04                       0                36
## 6                2.51                       0                38
##   FairlyActiveMinutes LightlyActiveMinutes SedentaryMinutes Calories
## 1                  13                  328              728     1985
## 2                  19                  217              776     1797
## 3                  11                  181             1218     1776
## 4                  34                  209              726     1745
## 5                  10                  221              773     1863
## 6                  20                  164              539     1728
\end{verbatim}

\begin{Shaded}
\begin{Highlighting}[]
\FunctionTok{str}\NormalTok{(daily\_activity)}
\end{Highlighting}
\end{Shaded}

\begin{verbatim}
## 'data.frame':    940 obs. of  15 variables:
##  $ Id                      : num  1.5e+09 1.5e+09 1.5e+09 1.5e+09 1.5e+09 ...
##  $ ActivityDate            : chr  "4/12/2016" "4/13/2016" "4/14/2016" "4/15/2016" ...
##  $ TotalSteps              : int  13162 10735 10460 9762 12669 9705 13019 15506 10544 9819 ...
##  $ TotalDistance           : num  8.5 6.97 6.74 6.28 8.16 ...
##  $ TrackerDistance         : num  8.5 6.97 6.74 6.28 8.16 ...
##  $ LoggedActivitiesDistance: num  0 0 0 0 0 0 0 0 0 0 ...
##  $ VeryActiveDistance      : num  1.88 1.57 2.44 2.14 2.71 ...
##  $ ModeratelyActiveDistance: num  0.55 0.69 0.4 1.26 0.41 ...
##  $ LightActiveDistance     : num  6.06 4.71 3.91 2.83 5.04 ...
##  $ SedentaryActiveDistance : num  0 0 0 0 0 0 0 0 0 0 ...
##  $ VeryActiveMinutes       : int  25 21 30 29 36 38 42 50 28 19 ...
##  $ FairlyActiveMinutes     : int  13 19 11 34 10 20 16 31 12 8 ...
##  $ LightlyActiveMinutes    : int  328 217 181 209 221 164 233 264 205 211 ...
##  $ SedentaryMinutes        : int  728 776 1218 726 773 539 1149 775 818 838 ...
##  $ Calories                : int  1985 1797 1776 1745 1863 1728 1921 2035 1786 1775 ...
\end{verbatim}

\hypertarget{cleaning-and-formatting}{%
\subsubsection{CLEANING AND FORMATTING}\label{cleaning-and-formatting}}

Now that we got to know more about our data structures we will process
them to look for any errors and inconsistencies.

\hypertarget{verifying-number-of-users}{%
\paragraph{VERIFYING NUMBER OF USERS}\label{verifying-number-of-users}}

Before we continue with our cleaning we want to make sure how many
unique users are per data frame.

\begin{Shaded}
\begin{Highlighting}[]
\FunctionTok{n\_unique}\NormalTok{(daily\_steps}\SpecialCharTok{$}\NormalTok{Id)}
\end{Highlighting}
\end{Shaded}

\begin{verbatim}
## [1] 33
\end{verbatim}

\begin{Shaded}
\begin{Highlighting}[]
\FunctionTok{n\_unique}\NormalTok{(daily\_intensities}\SpecialCharTok{$}\NormalTok{Id)}
\end{Highlighting}
\end{Shaded}

\begin{verbatim}
## [1] 33
\end{verbatim}

\begin{Shaded}
\begin{Highlighting}[]
\FunctionTok{n\_unique}\NormalTok{(daily\_calories}\SpecialCharTok{$}\NormalTok{Id)}
\end{Highlighting}
\end{Shaded}

\begin{verbatim}
## [1] 33
\end{verbatim}

\begin{Shaded}
\begin{Highlighting}[]
\FunctionTok{n\_unique}\NormalTok{(hourly\_steps}\SpecialCharTok{$}\NormalTok{Id)}
\end{Highlighting}
\end{Shaded}

\begin{verbatim}
## [1] 33
\end{verbatim}

\begin{Shaded}
\begin{Highlighting}[]
\FunctionTok{n\_unique}\NormalTok{(daily\_activity}\SpecialCharTok{$}\NormalTok{Id)}
\end{Highlighting}
\end{Shaded}

\begin{verbatim}
## [1] 33
\end{verbatim}

\hypertarget{duplicates}{%
\paragraph{DUPLICATES}\label{duplicates}}

We will now look for any duplicates

\begin{Shaded}
\begin{Highlighting}[]
\FunctionTok{sum}\NormalTok{(}\FunctionTok{duplicated}\NormalTok{(daily\_steps))}
\end{Highlighting}
\end{Shaded}

\begin{verbatim}
## [1] 0
\end{verbatim}

\begin{Shaded}
\begin{Highlighting}[]
\FunctionTok{sum}\NormalTok{(}\FunctionTok{duplicated}\NormalTok{(daily\_calories))}
\end{Highlighting}
\end{Shaded}

\begin{verbatim}
## [1] 0
\end{verbatim}

\begin{Shaded}
\begin{Highlighting}[]
\FunctionTok{sum}\NormalTok{(}\FunctionTok{duplicated}\NormalTok{(daily\_intensities))}
\end{Highlighting}
\end{Shaded}

\begin{verbatim}
## [1] 0
\end{verbatim}

\begin{Shaded}
\begin{Highlighting}[]
\FunctionTok{sum}\NormalTok{(}\FunctionTok{duplicated}\NormalTok{(hourly\_steps))}
\end{Highlighting}
\end{Shaded}

\begin{verbatim}
## [1] 0
\end{verbatim}

\begin{Shaded}
\begin{Highlighting}[]
\FunctionTok{sum}\NormalTok{(}\FunctionTok{duplicated}\NormalTok{(daily\_activity))}
\end{Highlighting}
\end{Shaded}

\begin{verbatim}
## [1] 0
\end{verbatim}

\hypertarget{clean-and-rename-columns}{%
\paragraph{CLEAN AND RENAME COLUMNS}\label{clean-and-rename-columns}}

We want to ensure that column names are using right syntax and same
format in all datasets since we will merge them later on. We are
changing the format of all columns to lower case.

\begin{Shaded}
\begin{Highlighting}[]
\FunctionTok{clean\_names}\NormalTok{(daily\_steps)}
\NormalTok{daily\_steps}\OtherTok{\textless{}{-}} \FunctionTok{rename\_with}\NormalTok{(daily\_steps, tolower)}

\FunctionTok{clean\_names}\NormalTok{(daily\_calories)}
\NormalTok{daily\_calories }\OtherTok{\textless{}{-}} \FunctionTok{rename\_with}\NormalTok{(daily\_calories, tolower)}

\FunctionTok{clean\_names}\NormalTok{(daily\_intensities)}
\NormalTok{daily\_intensities}\OtherTok{\textless{}{-}} \FunctionTok{rename\_with}\NormalTok{(daily\_intensities, tolower)}

\FunctionTok{clean\_names}\NormalTok{(hourly\_steps)}
\NormalTok{hourly\_steps }\OtherTok{\textless{}{-}} \FunctionTok{rename\_with}\NormalTok{(hourly\_steps, tolower)}

\FunctionTok{clean\_names}\NormalTok{(daily\_activity)}
\NormalTok{daily\_activity }\OtherTok{\textless{}{-}} \FunctionTok{rename\_with}\NormalTok{(daily\_activity, tolower)}

\NormalTok{products }\OtherTok{\textless{}{-}} \FunctionTok{rename\_with}\NormalTok{(products, tolower)}
\end{Highlighting}
\end{Shaded}

\hypertarget{consistency-of-columns}{%
\paragraph{CONSISTENCY OF COLUMNS}\label{consistency-of-columns}}

Make sure the column names are consistent across the files used and
check date format.

\begin{Shaded}
\begin{Highlighting}[]
\NormalTok{hourly\_steps }\OtherTok{\textless{}{-}}\NormalTok{ hourly\_steps }\SpecialCharTok{\%\textgreater{}\%}
  \FunctionTok{rename}\NormalTok{(}\AttributeTok{date\_time =}\NormalTok{ activityhour) }\SpecialCharTok{\%\textgreater{}\%}
  \FunctionTok{mutate}\NormalTok{(}\AttributeTok{date\_time =} \FunctionTok{as.POSIXct}\NormalTok{(date\_time, }\AttributeTok{format =} \StringTok{"\%m/\%d/\%Y \%I:\%M:\%S \%p"}\NormalTok{))}
\FunctionTok{head}\NormalTok{(hourly\_steps)}
\end{Highlighting}
\end{Shaded}

\begin{verbatim}
##           id           date_time steptotal
## 1 1503960366 2016-04-12 00:00:00       373
## 2 1503960366 2016-04-12 01:00:00       160
## 3 1503960366 2016-04-12 02:00:00       151
## 4 1503960366 2016-04-12 03:00:00         0
## 5 1503960366 2016-04-12 04:00:00         0
## 6 1503960366 2016-04-12 05:00:00         0
\end{verbatim}

\hypertarget{merging-datasets}{%
\paragraph{MERGING DATASETS}\label{merging-datasets}}

We will merge daily\_intensities and daily\_steps with daily\_calories
to see correlation between variables by using id as their primary keys.

\begin{Shaded}
\begin{Highlighting}[]
\NormalTok{user\_dailyx }\OtherTok{\textless{}{-}} \FunctionTok{merge}\NormalTok{(daily\_intensities,daily\_calories ,}\AttributeTok{by =}\FunctionTok{c}\NormalTok{ (}\StringTok{"id"}\NormalTok{,}\StringTok{"activityday"}\NormalTok{))}
\NormalTok{user\_dailyx}\SpecialCharTok{$}\NormalTok{activityday }\OtherTok{\textless{}{-}} \FunctionTok{as.Date}\NormalTok{(user\_dailyx}\SpecialCharTok{$}\NormalTok{activityday, }\AttributeTok{format =} \StringTok{"\%m/\%d/\%Y"}\NormalTok{)}
\FunctionTok{glimpse}\NormalTok{(user\_dailyx)}
\end{Highlighting}
\end{Shaded}

\begin{verbatim}
## Rows: 940
## Columns: 11
## $ id                       <dbl> 1503960366, 1503960366, 1503960366, 150396036~
## $ activityday              <date> 2016-04-12, 2016-04-13, 2016-04-14, 2016-04-~
## $ sedentaryminutes         <int> 728, 776, 1218, 726, 773, 539, 1149, 775, 818~
## $ lightlyactiveminutes     <int> 328, 217, 181, 209, 221, 164, 233, 264, 205, ~
## $ fairlyactiveminutes      <int> 13, 19, 11, 34, 10, 20, 16, 31, 12, 8, 27, 21~
## $ veryactiveminutes        <int> 25, 21, 30, 29, 36, 38, 42, 50, 28, 19, 66, 4~
## $ sedentaryactivedistance  <dbl> 0, 0, 0, 0, 0, 0, 0, 0, 0, 0, 0, 0, 0, 0, 0, ~
## $ lightactivedistance      <dbl> 6.06, 4.71, 3.91, 2.83, 5.04, 2.51, 4.71, 5.0~
## $ moderatelyactivedistance <dbl> 0.55, 0.69, 0.40, 1.26, 0.41, 0.78, 0.64, 1.3~
## $ veryactivedistance       <dbl> 1.88, 1.57, 2.44, 2.14, 2.71, 3.19, 3.25, 3.5~
## $ calories                 <int> 1985, 1797, 1776, 1745, 1863, 1728, 1921, 203~
\end{verbatim}

\begin{Shaded}
\begin{Highlighting}[]
\NormalTok{user\_dailyy }\OtherTok{\textless{}{-}} \FunctionTok{merge}\NormalTok{(daily\_steps ,daily\_calories ,}\AttributeTok{by =}\FunctionTok{c}\NormalTok{ (}\StringTok{"id"}\NormalTok{,}\StringTok{"activityday"}\NormalTok{))}

\NormalTok{user\_dailyy}\SpecialCharTok{$}\NormalTok{activityday }\OtherTok{\textless{}{-}} \FunctionTok{as.Date}\NormalTok{(user\_dailyy}\SpecialCharTok{$}\NormalTok{activityday, }\AttributeTok{format =} \StringTok{"\%m/\%d/\%Y"}\NormalTok{)}
\FunctionTok{glimpse}\NormalTok{(user\_dailyy)}
\end{Highlighting}
\end{Shaded}

\begin{verbatim}
## Rows: 940
## Columns: 4
## $ id          <dbl> 1503960366, 1503960366, 1503960366, 1503960366, 1503960366~
## $ activityday <date> 2016-04-12, 2016-04-13, 2016-04-14, 2016-04-15, 2016-04-1~
## $ steptotal   <int> 13162, 10735, 10460, 9762, 12669, 9705, 13019, 15506, 1054~
## $ calories    <int> 1985, 1797, 1776, 1745, 1863, 1728, 1921, 2035, 1786, 1775~
\end{verbatim}

\begin{Shaded}
\begin{Highlighting}[]
\NormalTok{user\_dailyz }\OtherTok{\textless{}{-}} \FunctionTok{merge}\NormalTok{(daily\_intensities ,daily\_steps ,}\AttributeTok{by =}\FunctionTok{c}\NormalTok{ (}\StringTok{"id"}\NormalTok{,}\StringTok{"activityday"}\NormalTok{))}
\NormalTok{user\_dailyz}\SpecialCharTok{$}\NormalTok{activityday }\OtherTok{\textless{}{-}} \FunctionTok{as.Date}\NormalTok{(user\_dailyz}\SpecialCharTok{$}\NormalTok{activityday, }\AttributeTok{format =} \StringTok{"\%m/\%d/\%Y"}\NormalTok{)}
\FunctionTok{glimpse}\NormalTok{(user\_dailyz)}
\end{Highlighting}
\end{Shaded}

\begin{verbatim}
## Rows: 940
## Columns: 11
## $ id                       <dbl> 1503960366, 1503960366, 1503960366, 150396036~
## $ activityday              <date> 2016-04-12, 2016-04-13, 2016-04-14, 2016-04-~
## $ sedentaryminutes         <int> 728, 776, 1218, 726, 773, 539, 1149, 775, 818~
## $ lightlyactiveminutes     <int> 328, 217, 181, 209, 221, 164, 233, 264, 205, ~
## $ fairlyactiveminutes      <int> 13, 19, 11, 34, 10, 20, 16, 31, 12, 8, 27, 21~
## $ veryactiveminutes        <int> 25, 21, 30, 29, 36, 38, 42, 50, 28, 19, 66, 4~
## $ sedentaryactivedistance  <dbl> 0, 0, 0, 0, 0, 0, 0, 0, 0, 0, 0, 0, 0, 0, 0, ~
## $ lightactivedistance      <dbl> 6.06, 4.71, 3.91, 2.83, 5.04, 2.51, 4.71, 5.0~
## $ moderatelyactivedistance <dbl> 0.55, 0.69, 0.40, 1.26, 0.41, 0.78, 0.64, 1.3~
## $ veryactivedistance       <dbl> 1.88, 1.57, 2.44, 2.14, 2.71, 3.19, 3.25, 3.5~
## $ steptotal                <int> 13162, 10735, 10460, 9762, 12669, 9705, 13019~
\end{verbatim}

\hypertarget{phase4-5analyze-share}{%
\subsection{PHASE4-5:ANALYZE-SHARE}\label{phase4-5analyze-share}}

\hypertarget{type-of-users}{%
\subsubsection{TYPE OF USERS}\label{type-of-users}}

We can classify the users by activity considering the daily amount of
steps. We can categorize users as follows

\begin{itemize}
\tightlist
\item
  Sedentary - Less than 5000 steps a day.
\item
  Lightly active - Between 5000 and 7499 steps a day.
\item
  Fairly active - Between 7500 and 9999 steps a day.
\item
  Very active - More than 10000 steps a day.
\end{itemize}

Classification has been made per the following article
\href{URL}{https://www.10000steps.org.au/articles/counting-steps/} First
we will calculate the daily steps average by user.

\begin{Shaded}
\begin{Highlighting}[]
\NormalTok{daily\_average }\OtherTok{\textless{}{-}}\NormalTok{ user\_dailyy }\SpecialCharTok{\%\textgreater{}\%}
  \FunctionTok{group\_by}\NormalTok{(id) }\SpecialCharTok{\%\textgreater{}\%}
  \FunctionTok{summarise}\NormalTok{ (}\AttributeTok{mean\_daily\_steps =} \FunctionTok{mean}\NormalTok{(steptotal), }\AttributeTok{mean\_daily\_calories =} \FunctionTok{mean}\NormalTok{(calories))}

\FunctionTok{head}\NormalTok{(daily\_average)}
\end{Highlighting}
\end{Shaded}

\begin{verbatim}
## # A tibble: 6 x 3
##           id mean_daily_steps mean_daily_calories
##        <dbl>            <dbl>               <dbl>
## 1 1503960366           12117.               1816.
## 2 1624580081            5744.               1483.
## 3 1644430081            7283.               2811.
## 4 1844505072            2580.               1573.
## 5 1927972279             916.               2173.
## 6 2022484408           11371.               2510.
\end{verbatim}

We will now classify users based on average daily steps

\begin{Shaded}
\begin{Highlighting}[]
\NormalTok{user\_type }\OtherTok{\textless{}{-}}\NormalTok{ daily\_average }\SpecialCharTok{\%\textgreater{}\%}
  \FunctionTok{mutate}\NormalTok{(}\AttributeTok{user\_type =} \FunctionTok{case\_when}\NormalTok{(}
\NormalTok{    mean\_daily\_steps }\SpecialCharTok{\textless{}} \DecValTok{5000} \SpecialCharTok{\textasciitilde{}} \StringTok{"sedentary"}\NormalTok{,}
\NormalTok{    mean\_daily\_steps }\SpecialCharTok{\textgreater{}=} \DecValTok{5000} \SpecialCharTok{\&}\NormalTok{ mean\_daily\_steps }\SpecialCharTok{\textless{}} \DecValTok{7499} \SpecialCharTok{\textasciitilde{}} \StringTok{"lightly active"}\NormalTok{, }
\NormalTok{    mean\_daily\_steps }\SpecialCharTok{\textgreater{}=} \DecValTok{7500} \SpecialCharTok{\&}\NormalTok{ mean\_daily\_steps }\SpecialCharTok{\textless{}} \DecValTok{9999} \SpecialCharTok{\textasciitilde{}} \StringTok{"fairly active"}\NormalTok{, }
\NormalTok{    mean\_daily\_steps }\SpecialCharTok{\textgreater{}=} \DecValTok{10000} \SpecialCharTok{\textasciitilde{}} \StringTok{"very active"}
\NormalTok{  ))}

\FunctionTok{head}\NormalTok{(user\_type)}
\end{Highlighting}
\end{Shaded}

\begin{verbatim}
## # A tibble: 6 x 4
##           id mean_daily_steps mean_daily_calories user_type     
##        <dbl>            <dbl>               <dbl> <chr>         
## 1 1503960366           12117.               1816. very active   
## 2 1624580081            5744.               1483. lightly active
## 3 1644430081            7283.               2811. lightly active
## 4 1844505072            2580.               1573. sedentary     
## 5 1927972279             916.               2173. sedentary     
## 6 2022484408           11371.               2510. very active
\end{verbatim}

Now that we have a new column with the user type we will create a data
frame with the percentage of each user type to better visualize them on
a graph.

\begin{Shaded}
\begin{Highlighting}[]
\NormalTok{user\_type\_percent }\OtherTok{\textless{}{-}}\NormalTok{ user\_type }\SpecialCharTok{\%\textgreater{}\%}
  \FunctionTok{group\_by}\NormalTok{(user\_type) }\SpecialCharTok{\%\textgreater{}\%}
  \FunctionTok{summarise}\NormalTok{(}\AttributeTok{total =} \FunctionTok{n}\NormalTok{()) }\SpecialCharTok{\%\textgreater{}\%}
  \FunctionTok{mutate}\NormalTok{(}\AttributeTok{totals =} \FunctionTok{sum}\NormalTok{(total)) }\SpecialCharTok{\%\textgreater{}\%}
  \FunctionTok{group\_by}\NormalTok{(user\_type) }\SpecialCharTok{\%\textgreater{}\%}
  \FunctionTok{summarise}\NormalTok{(}\AttributeTok{total\_percent =}\NormalTok{ total }\SpecialCharTok{/}\NormalTok{ totals) }\SpecialCharTok{\%\textgreater{}\%}
  \FunctionTok{mutate}\NormalTok{(}\AttributeTok{labels =}\NormalTok{ scales}\SpecialCharTok{::}\FunctionTok{percent}\NormalTok{(total\_percent))}

\NormalTok{user\_type\_percent}\SpecialCharTok{$}\NormalTok{user\_type }\OtherTok{\textless{}{-}} \FunctionTok{factor}\NormalTok{(user\_type\_percent}\SpecialCharTok{$}\NormalTok{user\_type , }\AttributeTok{levels =} \FunctionTok{c}\NormalTok{(}\StringTok{"very active"}\NormalTok{, }\StringTok{"fairly active"}\NormalTok{, }\StringTok{"lightly active"}\NormalTok{, }\StringTok{"sedentary"}\NormalTok{))}


\FunctionTok{head}\NormalTok{(user\_type\_percent)}
\end{Highlighting}
\end{Shaded}

\begin{verbatim}
## # A tibble: 4 x 3
##   user_type      total_percent labels
##   <fct>                  <dbl> <chr> 
## 1 fairly active          0.273 27.3% 
## 2 lightly active         0.273 27.3% 
## 3 sedentary              0.242 24.2% 
## 4 very active            0.212 21.2%
\end{verbatim}

Below we can see that users are fairly distributed by their activity
considering the daily amount of steps. We can determine that based on
users activity all kind of users use smart-devices.

\begin{Shaded}
\begin{Highlighting}[]
\NormalTok{user\_type\_percent }\SpecialCharTok{\%\textgreater{}\%}
  \FunctionTok{ggplot}\NormalTok{(}\FunctionTok{aes}\NormalTok{(}\AttributeTok{x=}\StringTok{""}\NormalTok{,}\AttributeTok{y=}\NormalTok{total\_percent, }\AttributeTok{fill=}\NormalTok{user\_type)) }\SpecialCharTok{+}
  \FunctionTok{geom\_bar}\NormalTok{(}\AttributeTok{stat =} \StringTok{"identity"}\NormalTok{, }\AttributeTok{width =} \DecValTok{1}\NormalTok{)}\SpecialCharTok{+}
  \FunctionTok{coord\_polar}\NormalTok{(}\StringTok{"y"}\NormalTok{, }\AttributeTok{start=}\DecValTok{0}\NormalTok{)}\SpecialCharTok{+}
  \FunctionTok{theme\_minimal}\NormalTok{()}\SpecialCharTok{+}
  \FunctionTok{theme}\NormalTok{(}\AttributeTok{axis.title.x=} \FunctionTok{element\_blank}\NormalTok{(),}
        \AttributeTok{axis.title.y =} \FunctionTok{element\_blank}\NormalTok{(),}
        \AttributeTok{panel.border =} \FunctionTok{element\_blank}\NormalTok{(), }
        \AttributeTok{panel.grid =} \FunctionTok{element\_blank}\NormalTok{(), }
        \AttributeTok{axis.ticks =} \FunctionTok{element\_blank}\NormalTok{(),}
        \AttributeTok{axis.text.x =} \FunctionTok{element\_blank}\NormalTok{(),}
        \AttributeTok{plot.title =} \FunctionTok{element\_text}\NormalTok{(}\AttributeTok{hjust =} \FloatTok{0.5}\NormalTok{, }\AttributeTok{size=}\DecValTok{14}\NormalTok{, }\AttributeTok{face =} \StringTok{"bold"}\NormalTok{)) }\SpecialCharTok{+}
  \FunctionTok{scale\_fill\_manual}\NormalTok{(}\AttributeTok{values =} \FunctionTok{c}\NormalTok{(}\StringTok{"\#85e085"}\NormalTok{,}\StringTok{"\#e6e600"}\NormalTok{, }\StringTok{"\#ffd480"}\NormalTok{, }\StringTok{"\#ff8080"}\NormalTok{)) }\SpecialCharTok{+}
  \FunctionTok{geom\_text}\NormalTok{(}\FunctionTok{aes}\NormalTok{(}\AttributeTok{label =}\NormalTok{ labels),}
            \AttributeTok{position =} \FunctionTok{position\_stack}\NormalTok{(}\AttributeTok{vjust =} \FloatTok{0.5}\NormalTok{))}\SpecialCharTok{+}
  \FunctionTok{labs}\NormalTok{(}\AttributeTok{title=}\StringTok{"User type distribution"}\NormalTok{)}
\end{Highlighting}
\end{Shaded}

\includegraphics{bellabeat_notebook_files/figure-latex/unnamed-chunk-15-1.pdf}

\hypertarget{steps-per-days-of-week}{%
\subsubsection{STEPS PER DAYS OF WEEK}\label{steps-per-days-of-week}}

We want to know now what days of the week are the users more active. We
will also verify if the users walk the recommended amount of steps.
Below we are calculating the weekdays based on our column date. We are
also calculating the average steps walked by days of week.

\begin{Shaded}
\begin{Highlighting}[]
\NormalTok{weekday\_steps }\OtherTok{\textless{}{-}}\NormalTok{ user\_dailyy }\SpecialCharTok{\%\textgreater{}\%}
  \FunctionTok{mutate}\NormalTok{(}\AttributeTok{weekday =} \FunctionTok{weekdays}\NormalTok{(activityday))}

\NormalTok{weekday\_steps}\SpecialCharTok{$}\NormalTok{weekday }\OtherTok{\textless{}{-}}\FunctionTok{ordered}\NormalTok{(weekday\_steps}\SpecialCharTok{$}\NormalTok{weekday, }\AttributeTok{levels=}\FunctionTok{c}\NormalTok{(}\StringTok{"Monday"}\NormalTok{, }\StringTok{"Tuesday"}\NormalTok{, }\StringTok{"Wednesday"}\NormalTok{, }\StringTok{"Thursday"}\NormalTok{,}
\StringTok{"Friday"}\NormalTok{, }\StringTok{"Saturday"}\NormalTok{, }\StringTok{"Sunday"}\NormalTok{))}

\NormalTok{weekday\_steps}\OtherTok{\textless{}{-}}\NormalTok{weekday\_steps}\SpecialCharTok{\%\textgreater{}\%}
  \FunctionTok{group\_by}\NormalTok{(weekday) }\SpecialCharTok{\%\textgreater{}\%}
  \FunctionTok{summarize}\NormalTok{ (}\AttributeTok{user\_dailyy=} \FunctionTok{mean}\NormalTok{(steptotal))}

\FunctionTok{head}\NormalTok{(weekday\_steps)}
\end{Highlighting}
\end{Shaded}

\begin{verbatim}
## # A tibble: 6 x 2
##   weekday   user_dailyy
##   <ord>           <dbl>
## 1 Monday          7781.
## 2 Tuesday         8125.
## 3 Wednesday       7559.
## 4 Thursday        7406.
## 5 Friday          7448.
## 6 Saturday        8153.
\end{verbatim}

\begin{Shaded}
\begin{Highlighting}[]
\FunctionTok{ggarrange}\NormalTok{(}
    \FunctionTok{ggplot}\NormalTok{(weekday\_steps) }\SpecialCharTok{+}
      \FunctionTok{geom\_col}\NormalTok{(}\FunctionTok{aes}\NormalTok{(weekday, user\_dailyy), }\AttributeTok{fill =} \StringTok{"\#006699"}\NormalTok{) }\SpecialCharTok{+}
      \FunctionTok{geom\_hline}\NormalTok{(}\AttributeTok{yintercept =} \DecValTok{7500}\NormalTok{) }\SpecialCharTok{+}
      \FunctionTok{labs}\NormalTok{(}\AttributeTok{title =} \StringTok{"Daily steps per weekday"}\NormalTok{, }\AttributeTok{x=} \StringTok{""}\NormalTok{, }\AttributeTok{y =} \StringTok{""}\NormalTok{) }\SpecialCharTok{+}
      \FunctionTok{theme}\NormalTok{(}\AttributeTok{axis.text.x =} \FunctionTok{element\_text}\NormalTok{(}\AttributeTok{angle =} \DecValTok{45}\NormalTok{,}\AttributeTok{vjust =} \FloatTok{0.5}\NormalTok{, }\AttributeTok{hjust =} \DecValTok{1}\NormalTok{))}
\NormalTok{)}
\end{Highlighting}
\end{Shaded}

\includegraphics{bellabeat_notebook_files/figure-latex/unnamed-chunk-17-1.pdf}

In the graph above we can determine that users mostly walk daily the
recommended amount of steps of 7500 besides holdidays(Sunday).

\hypertarget{user-activity}{%
\subsubsection{USER ACTIVITY}\label{user-activity}}

Getting deeper into our analysis we want to know when exactly are users
more active in a day.

We will use the hourly\_steps data frame and separate activityhour
column.

\begin{Shaded}
\begin{Highlighting}[]
\NormalTok{hourly\_steps }\OtherTok{\textless{}{-}}\NormalTok{ hourly\_steps }\SpecialCharTok{\%\textgreater{}\%}
  \FunctionTok{separate}\NormalTok{(date\_time, }\AttributeTok{into =} \FunctionTok{c}\NormalTok{(}\StringTok{"activityday"}\NormalTok{, }\StringTok{"time"}\NormalTok{), }\AttributeTok{sep =} \StringTok{" "}\NormalTok{)}\SpecialCharTok{\%\textgreater{}\%}
  \FunctionTok{mutate}\NormalTok{(}\AttributeTok{activityday =} \FunctionTok{parse\_date}\NormalTok{(activityday))}
\end{Highlighting}
\end{Shaded}

\begin{verbatim}
## Warning: Expected 2 pieces. Missing pieces filled with `NA` in 934 rows [1, 25, 49, 73,
## 97, 121, 145, 169, 193, 217, 241, 265, 289, 313, 337, 361, 385, 409, 433, 457,
## ...].
\end{verbatim}

\begin{Shaded}
\begin{Highlighting}[]
\NormalTok{hourly\_steps}\SpecialCharTok{$}\NormalTok{time[}\FunctionTok{is.na}\NormalTok{(hourly\_steps}\SpecialCharTok{$}\NormalTok{time)] }\OtherTok{\textless{}{-}} \StringTok{"00:00:00"}
\FunctionTok{head}\NormalTok{(hourly\_steps)}
\end{Highlighting}
\end{Shaded}

\begin{verbatim}
##           id activityday     time steptotal
## 1 1503960366  2016-04-12 00:00:00       373
## 2 1503960366  2016-04-12 01:00:00       160
## 3 1503960366  2016-04-12 02:00:00       151
## 4 1503960366  2016-04-12 03:00:00         0
## 5 1503960366  2016-04-12 04:00:00         0
## 6 1503960366  2016-04-12 05:00:00         0
\end{verbatim}

\begin{Shaded}
\begin{Highlighting}[]
\NormalTok{hourly\_steps }\SpecialCharTok{\%\textgreater{}\%}
  \FunctionTok{group\_by}\NormalTok{(time) }\SpecialCharTok{\%\textgreater{}\%}
  \FunctionTok{summarize}\NormalTok{(}\AttributeTok{average\_steps =} \FunctionTok{mean}\NormalTok{(steptotal)) }\SpecialCharTok{\%\textgreater{}\%}
  \FunctionTok{ggplot}\NormalTok{() }\SpecialCharTok{+}
  \FunctionTok{geom\_col}\NormalTok{(}\AttributeTok{mapping =} \FunctionTok{aes}\NormalTok{(}\AttributeTok{x=}\NormalTok{time, }\AttributeTok{y =}\NormalTok{ average\_steps, }\AttributeTok{fill =}\NormalTok{ average\_steps)) }\SpecialCharTok{+} 
  \FunctionTok{labs}\NormalTok{(}\AttributeTok{title =} \StringTok{"Hourly steps throughout the day"}\NormalTok{, }\AttributeTok{x=}\StringTok{""}\NormalTok{, }\AttributeTok{y=}\StringTok{""}\NormalTok{) }\SpecialCharTok{+} 
  \FunctionTok{scale\_fill\_gradient}\NormalTok{(}\AttributeTok{low =} \StringTok{"green"}\NormalTok{, }\AttributeTok{high =} \StringTok{"red"}\NormalTok{)}\SpecialCharTok{+}
  \FunctionTok{theme}\NormalTok{(}\AttributeTok{axis.text.x =} \FunctionTok{element\_text}\NormalTok{(}\AttributeTok{angle =} \DecValTok{90}\NormalTok{))}
\end{Highlighting}
\end{Shaded}

\includegraphics{bellabeat_notebook_files/figure-latex/unnamed-chunk-19-1.pdf}

We can see that users are more active between 8am and 7pm walking more
steps during lunch time from 12pm to 2pm and evenings from 5pm and 7pm.

\hypertarget{correlations}{%
\subsubsection{CORRELATIONS}\label{correlations}}

We will now determine if there is any correlation between different
variables: *Daily steps and calories

\begin{Shaded}
\begin{Highlighting}[]
\FunctionTok{ggarrange}\NormalTok{(}
\FunctionTok{ggplot}\NormalTok{(user\_dailyy, }\FunctionTok{aes}\NormalTok{(}\AttributeTok{x=}\NormalTok{steptotal, }\AttributeTok{y=}\NormalTok{calories))}\SpecialCharTok{+}
  \FunctionTok{geom\_jitter}\NormalTok{() }\SpecialCharTok{+}
  \FunctionTok{geom\_smooth}\NormalTok{(}\AttributeTok{color =} \StringTok{"red"}\NormalTok{) }\SpecialCharTok{+} 
  \FunctionTok{labs}\NormalTok{(}\AttributeTok{title =} \StringTok{"Daily steps vs Calories"}\NormalTok{, }\AttributeTok{x =} \StringTok{"Daily steps"}\NormalTok{, }\AttributeTok{y=} \StringTok{"Calories"}\NormalTok{) }\SpecialCharTok{+}
   \FunctionTok{theme}\NormalTok{(}\AttributeTok{panel.background =} \FunctionTok{element\_blank}\NormalTok{(),}
        \AttributeTok{plot.title =} \FunctionTok{element\_text}\NormalTok{( }\AttributeTok{size=}\DecValTok{14}\NormalTok{))}
\NormalTok{)}
\end{Highlighting}
\end{Shaded}

\begin{verbatim}
## `geom_smooth()` using method = 'loess' and formula = 'y ~ x'
\end{verbatim}

\includegraphics{bellabeat_notebook_files/figure-latex/unnamed-chunk-20-1.pdf}

We can see a positive correlation between steps and calories burned. As
assumed the more steps walked the more calories may be burned.

\hypertarget{use-of-smart-device}{%
\subsubsection{USE OF SMART DEVICE}\label{use-of-smart-device}}

\hypertarget{days-used-smart-device}{%
\paragraph{Days used smart device}\label{days-used-smart-device}}

We will calculate the number of users that use their smart device on a
daily basis, classifying our sample into three categories knowing that
the date interval is 31 days

\begin{itemize}
\tightlist
\item
  high use - users who use their device between 21 and 31 days.
\item
  moderate use - users who use their device between 10 and 20 days.
\item
  low use - users who use their device between 1 and 10 days.
\end{itemize}

First we will create a new data frame grouping by Id, calculating number
of days used and creating a new column with the classification explained
above.

\begin{Shaded}
\begin{Highlighting}[]
\NormalTok{daily\_use }\OtherTok{\textless{}{-}}\NormalTok{ user\_dailyy }\SpecialCharTok{\%\textgreater{}\%}
  \FunctionTok{group\_by}\NormalTok{(id) }\SpecialCharTok{\%\textgreater{}\%}
  \FunctionTok{summarize}\NormalTok{(}\AttributeTok{days\_used=}\FunctionTok{sum}\NormalTok{(}\FunctionTok{n}\NormalTok{())) }\SpecialCharTok{\%\textgreater{}\%}
  \FunctionTok{mutate}\NormalTok{(}\AttributeTok{usage =} \FunctionTok{case\_when}\NormalTok{(}
\NormalTok{    days\_used }\SpecialCharTok{\textgreater{}=} \DecValTok{1} \SpecialCharTok{\&}\NormalTok{ days\_used }\SpecialCharTok{\textless{}=} \DecValTok{10} \SpecialCharTok{\textasciitilde{}} \StringTok{"low use"}\NormalTok{,}
\NormalTok{    days\_used }\SpecialCharTok{\textgreater{}=} \DecValTok{11} \SpecialCharTok{\&}\NormalTok{ days\_used }\SpecialCharTok{\textless{}=} \DecValTok{20} \SpecialCharTok{\textasciitilde{}} \StringTok{"moderate use"}\NormalTok{, }
\NormalTok{    days\_used }\SpecialCharTok{\textgreater{}=} \DecValTok{21} \SpecialCharTok{\&}\NormalTok{ days\_used }\SpecialCharTok{\textless{}=} \DecValTok{31} \SpecialCharTok{\textasciitilde{}} \StringTok{"high use"}\NormalTok{, }
\NormalTok{  ))}
  
\FunctionTok{head}\NormalTok{(daily\_use)}
\end{Highlighting}
\end{Shaded}

\begin{verbatim}
## # A tibble: 6 x 3
##           id days_used usage   
##        <dbl>     <int> <chr>   
## 1 1503960366        31 high use
## 2 1624580081        31 high use
## 3 1644430081        30 high use
## 4 1844505072        31 high use
## 5 1927972279        31 high use
## 6 2022484408        31 high use
\end{verbatim}

We will now create a percentage data frame to better visualize the
results in the graph. We are also ordering our usage levels.

\begin{Shaded}
\begin{Highlighting}[]
\NormalTok{daily\_use\_percent }\OtherTok{\textless{}{-}}\NormalTok{ daily\_use }\SpecialCharTok{\%\textgreater{}\%}
  \FunctionTok{group\_by}\NormalTok{(usage) }\SpecialCharTok{\%\textgreater{}\%}
  \FunctionTok{summarise}\NormalTok{(}\AttributeTok{total =} \FunctionTok{n}\NormalTok{()) }\SpecialCharTok{\%\textgreater{}\%}
  \FunctionTok{mutate}\NormalTok{(}\AttributeTok{totals =} \FunctionTok{sum}\NormalTok{(total)) }\SpecialCharTok{\%\textgreater{}\%}
  \FunctionTok{group\_by}\NormalTok{(usage) }\SpecialCharTok{\%\textgreater{}\%}
  \FunctionTok{summarise}\NormalTok{(}\AttributeTok{total\_percent =}\NormalTok{ total }\SpecialCharTok{/}\NormalTok{ totals) }\SpecialCharTok{\%\textgreater{}\%}
  \FunctionTok{mutate}\NormalTok{(}\AttributeTok{labels =}\NormalTok{ scales}\SpecialCharTok{::}\FunctionTok{percent}\NormalTok{(total\_percent))}

\NormalTok{daily\_use\_percent}\SpecialCharTok{$}\NormalTok{usage }\OtherTok{\textless{}{-}} \FunctionTok{factor}\NormalTok{(daily\_use\_percent}\SpecialCharTok{$}\NormalTok{usage, }\AttributeTok{levels =} \FunctionTok{c}\NormalTok{(}\StringTok{"high use"}\NormalTok{, }\StringTok{"moderate use"}\NormalTok{, }\StringTok{"low use"}\NormalTok{))}

\FunctionTok{head}\NormalTok{(daily\_use\_percent)}
\end{Highlighting}
\end{Shaded}

\begin{verbatim}
## # A tibble: 3 x 3
##   usage        total_percent labels
##   <fct>                <dbl> <chr> 
## 1 high use            0.879  87.9% 
## 2 low use             0.0303 3.0%  
## 3 moderate use        0.0909 9.1%
\end{verbatim}

\begin{Shaded}
\begin{Highlighting}[]
\NormalTok{daily\_use\_percent }\SpecialCharTok{\%\textgreater{}\%}
  \FunctionTok{ggplot}\NormalTok{(}\FunctionTok{aes}\NormalTok{(}\AttributeTok{x=}\StringTok{""}\NormalTok{,}\AttributeTok{y=}\NormalTok{total\_percent, }\AttributeTok{fill=}\NormalTok{usage)) }\SpecialCharTok{+}
  \FunctionTok{geom\_bar}\NormalTok{(}\AttributeTok{stat =} \StringTok{"identity"}\NormalTok{, }\AttributeTok{width =} \DecValTok{1}\NormalTok{)}\SpecialCharTok{+}
  \FunctionTok{coord\_polar}\NormalTok{(}\StringTok{"y"}\NormalTok{, }\AttributeTok{start=}\DecValTok{0}\NormalTok{)}\SpecialCharTok{+}
  \FunctionTok{theme\_minimal}\NormalTok{()}\SpecialCharTok{+}
  \FunctionTok{theme}\NormalTok{(}\AttributeTok{axis.title.x=} \FunctionTok{element\_blank}\NormalTok{(),}
        \AttributeTok{axis.title.y =} \FunctionTok{element\_blank}\NormalTok{(),}
        \AttributeTok{panel.border =} \FunctionTok{element\_blank}\NormalTok{(), }
        \AttributeTok{panel.grid =} \FunctionTok{element\_blank}\NormalTok{(), }
        \AttributeTok{axis.ticks =} \FunctionTok{element\_blank}\NormalTok{(),}
        \AttributeTok{axis.text.x =} \FunctionTok{element\_blank}\NormalTok{(),}
        \AttributeTok{plot.title =} \FunctionTok{element\_text}\NormalTok{(}\AttributeTok{hjust =} \FloatTok{0.5}\NormalTok{, }\AttributeTok{size=}\DecValTok{14}\NormalTok{, }\AttributeTok{face =} \StringTok{"bold"}\NormalTok{)) }\SpecialCharTok{+}
  \FunctionTok{geom\_text}\NormalTok{(}\FunctionTok{aes}\NormalTok{(}\AttributeTok{label =}\NormalTok{ labels),}
            \AttributeTok{position =} \FunctionTok{position\_stack}\NormalTok{(}\AttributeTok{vjust =} \FloatTok{0.5}\NormalTok{))}\SpecialCharTok{+}
  \FunctionTok{scale\_fill\_manual}\NormalTok{(}\AttributeTok{values =} \FunctionTok{c}\NormalTok{(}\StringTok{"\#006633"}\NormalTok{,}\StringTok{"\#00e673"}\NormalTok{,}\StringTok{"\#80ffbf"}\NormalTok{),}
                    \AttributeTok{labels =} \FunctionTok{c}\NormalTok{(}\StringTok{"High use {-} 21 to 31 days"}\NormalTok{,}
                                 \StringTok{"Moderate use {-} 11 to 20 days"}\NormalTok{,}
                                 \StringTok{"Low use {-} 1 to 10 days"}\NormalTok{))}\SpecialCharTok{+}
  \FunctionTok{labs}\NormalTok{(}\AttributeTok{title=}\StringTok{"Daily use of smart device"}\NormalTok{)}
\end{Highlighting}
\end{Shaded}

\includegraphics{bellabeat_notebook_files/figure-latex/unnamed-chunk-23-1.pdf}

Analyzing our results we can see that

\begin{itemize}
\tightlist
\item
  88\% of the users of our sample use their device frequently between 21
  to 31 days.
\item
  9\% use their device 11 to 20 days.
\item
  3\% of our sample use really rarely their device.
\end{itemize}

\hypertarget{time-used-smart-device}{%
\paragraph{Time used smart device}\label{time-used-smart-device}}

Being more precise we want to see how many minutes do users wear their
device per day. For that we will merge the created daily\_use data frame
and daily\_activity to be able to filter results by daily use of device
as well.

\begin{Shaded}
\begin{Highlighting}[]
\NormalTok{daily\_use\_merged }\OtherTok{\textless{}{-}} \FunctionTok{merge}\NormalTok{(daily\_activity, daily\_use, }\AttributeTok{by=}\FunctionTok{c}\NormalTok{ (}\StringTok{"id"}\NormalTok{))}
\FunctionTok{head}\NormalTok{(daily\_use\_merged)}
\end{Highlighting}
\end{Shaded}

\begin{verbatim}
##           id activitydate totalsteps totaldistance trackerdistance
## 1 1503960366    4/12/2016      13162          8.50            8.50
## 2 1503960366    4/13/2016      10735          6.97            6.97
## 3 1503960366    4/14/2016      10460          6.74            6.74
## 4 1503960366    4/15/2016       9762          6.28            6.28
## 5 1503960366    4/16/2016      12669          8.16            8.16
## 6 1503960366    4/17/2016       9705          6.48            6.48
##   loggedactivitiesdistance veryactivedistance moderatelyactivedistance
## 1                        0               1.88                     0.55
## 2                        0               1.57                     0.69
## 3                        0               2.44                     0.40
## 4                        0               2.14                     1.26
## 5                        0               2.71                     0.41
## 6                        0               3.19                     0.78
##   lightactivedistance sedentaryactivedistance veryactiveminutes
## 1                6.06                       0                25
## 2                4.71                       0                21
## 3                3.91                       0                30
## 4                2.83                       0                29
## 5                5.04                       0                36
## 6                2.51                       0                38
##   fairlyactiveminutes lightlyactiveminutes sedentaryminutes calories days_used
## 1                  13                  328              728     1985        31
## 2                  19                  217              776     1797        31
## 3                  11                  181             1218     1776        31
## 4                  34                  209              726     1745        31
## 5                  10                  221              773     1863        31
## 6                  20                  164              539     1728        31
##      usage
## 1 high use
## 2 high use
## 3 high use
## 4 high use
## 5 high use
## 6 high use
\end{verbatim}

We need to create a new data frame calculating the total amount of
minutes users wore the device every day and creating three different
categories

\begin{itemize}
\tightlist
\item
  All day - device was worn all day.
\item
  More than half day - device was worn more than half of the day.
\item
  Less than half day - device was worn less than half of the day.
\end{itemize}

\begin{Shaded}
\begin{Highlighting}[]
\NormalTok{minutes\_worn }\OtherTok{\textless{}{-}}\NormalTok{ daily\_use\_merged }\SpecialCharTok{\%\textgreater{}\%} 
  \FunctionTok{mutate}\NormalTok{(}\AttributeTok{total\_minutes\_worn =}\NormalTok{ veryactiveminutes}\SpecialCharTok{+}\NormalTok{fairlyactiveminutes}\SpecialCharTok{+}\NormalTok{lightlyactiveminutes}\SpecialCharTok{+}\NormalTok{sedentaryminutes)}\SpecialCharTok{\%\textgreater{}\%}
  \FunctionTok{mutate}\NormalTok{ (}\AttributeTok{percent\_minutes\_worn =}\NormalTok{ (total\_minutes\_worn}\SpecialCharTok{/}\DecValTok{1440}\NormalTok{)}\SpecialCharTok{*}\DecValTok{100}\NormalTok{) }\SpecialCharTok{\%\textgreater{}\%}
  \FunctionTok{mutate}\NormalTok{ (}\AttributeTok{worn =} \FunctionTok{case\_when}\NormalTok{(}
\NormalTok{    percent\_minutes\_worn }\SpecialCharTok{==} \DecValTok{100} \SpecialCharTok{\textasciitilde{}} \StringTok{"All day"}\NormalTok{,}
\NormalTok{    percent\_minutes\_worn }\SpecialCharTok{\textless{}} \DecValTok{100} \SpecialCharTok{\&}\NormalTok{ percent\_minutes\_worn }\SpecialCharTok{\textgreater{}=} \DecValTok{50}\SpecialCharTok{\textasciitilde{}} \StringTok{"More than half day"}\NormalTok{, }
\NormalTok{    percent\_minutes\_worn }\SpecialCharTok{\textless{}} \DecValTok{50} \SpecialCharTok{\&}\NormalTok{ percent\_minutes\_worn }\SpecialCharTok{\textgreater{}} \DecValTok{0} \SpecialCharTok{\textasciitilde{}} \StringTok{"Less than half day"}
\NormalTok{  ))}

\FunctionTok{head}\NormalTok{(minutes\_worn)}
\end{Highlighting}
\end{Shaded}

\begin{verbatim}
##           id activitydate totalsteps totaldistance trackerdistance
## 1 1503960366    4/12/2016      13162          8.50            8.50
## 2 1503960366    4/13/2016      10735          6.97            6.97
## 3 1503960366    4/14/2016      10460          6.74            6.74
## 4 1503960366    4/15/2016       9762          6.28            6.28
## 5 1503960366    4/16/2016      12669          8.16            8.16
## 6 1503960366    4/17/2016       9705          6.48            6.48
##   loggedactivitiesdistance veryactivedistance moderatelyactivedistance
## 1                        0               1.88                     0.55
## 2                        0               1.57                     0.69
## 3                        0               2.44                     0.40
## 4                        0               2.14                     1.26
## 5                        0               2.71                     0.41
## 6                        0               3.19                     0.78
##   lightactivedistance sedentaryactivedistance veryactiveminutes
## 1                6.06                       0                25
## 2                4.71                       0                21
## 3                3.91                       0                30
## 4                2.83                       0                29
## 5                5.04                       0                36
## 6                2.51                       0                38
##   fairlyactiveminutes lightlyactiveminutes sedentaryminutes calories days_used
## 1                  13                  328              728     1985        31
## 2                  19                  217              776     1797        31
## 3                  11                  181             1218     1776        31
## 4                  34                  209              726     1745        31
## 5                  10                  221              773     1863        31
## 6                  20                  164              539     1728        31
##      usage total_minutes_worn percent_minutes_worn               worn
## 1 high use               1094             75.97222 More than half day
## 2 high use               1033             71.73611 More than half day
## 3 high use               1440            100.00000            All day
## 4 high use                998             69.30556 More than half day
## 5 high use               1040             72.22222 More than half day
## 6 high use                761             52.84722 More than half day
\end{verbatim}

As we have done before, to better visualize our results we will create
new data frames. In this case we will create four different data frames
to arrange them later on on a same visualization.

\begin{itemize}
\tightlist
\item
  First data frame will show the total of users and will calculate
  percentage of minutes worn the device taking into consideration the
  three categories created.
\end{itemize}

\begin{Shaded}
\begin{Highlighting}[]
\NormalTok{minutes\_worn\_percent}\OtherTok{\textless{}{-}}\NormalTok{ minutes\_worn}\SpecialCharTok{\%\textgreater{}\%}
  \FunctionTok{group\_by}\NormalTok{(worn) }\SpecialCharTok{\%\textgreater{}\%}
  \FunctionTok{summarise}\NormalTok{(}\AttributeTok{total =} \FunctionTok{n}\NormalTok{()) }\SpecialCharTok{\%\textgreater{}\%}
  \FunctionTok{mutate}\NormalTok{(}\AttributeTok{totals =} \FunctionTok{sum}\NormalTok{(total)) }\SpecialCharTok{\%\textgreater{}\%}
  \FunctionTok{group\_by}\NormalTok{(worn) }\SpecialCharTok{\%\textgreater{}\%}
  \FunctionTok{summarise}\NormalTok{(}\AttributeTok{total\_percent =}\NormalTok{ total }\SpecialCharTok{/}\NormalTok{ totals) }\SpecialCharTok{\%\textgreater{}\%}
  \FunctionTok{mutate}\NormalTok{(}\AttributeTok{labels =}\NormalTok{ scales}\SpecialCharTok{::}\FunctionTok{percent}\NormalTok{(total\_percent))}
\FunctionTok{head}\NormalTok{(minutes\_worn\_percent)}
\end{Highlighting}
\end{Shaded}

\begin{verbatim}
## # A tibble: 3 x 3
##   worn               total_percent labels
##   <chr>                      <dbl> <chr> 
## 1 All day                   0.509  50.9% 
## 2 Less than half day        0.0266 2.7%  
## 3 More than half day        0.465  46.5%
\end{verbatim}

\begin{itemize}
\tightlist
\item
  The three other data frames are filtered by category of daily users so
  that we can see also the difference of daily use and time use.
\end{itemize}

\begin{Shaded}
\begin{Highlighting}[]
\NormalTok{minutes\_worn\_highuse }\OtherTok{\textless{}{-}}\NormalTok{ minutes\_worn}\SpecialCharTok{\%\textgreater{}\%}
  \FunctionTok{filter}\NormalTok{ (usage }\SpecialCharTok{==} \StringTok{"high use"}\NormalTok{)}\SpecialCharTok{\%\textgreater{}\%}
  \FunctionTok{group\_by}\NormalTok{(worn) }\SpecialCharTok{\%\textgreater{}\%}
  \FunctionTok{summarise}\NormalTok{(}\AttributeTok{total =} \FunctionTok{n}\NormalTok{()) }\SpecialCharTok{\%\textgreater{}\%}
  \FunctionTok{mutate}\NormalTok{(}\AttributeTok{totals =} \FunctionTok{sum}\NormalTok{(total)) }\SpecialCharTok{\%\textgreater{}\%}
  \FunctionTok{group\_by}\NormalTok{(worn) }\SpecialCharTok{\%\textgreater{}\%}
  \FunctionTok{summarise}\NormalTok{(}\AttributeTok{total\_percent =}\NormalTok{ total }\SpecialCharTok{/}\NormalTok{ totals) }\SpecialCharTok{\%\textgreater{}\%}
  \FunctionTok{mutate}\NormalTok{(}\AttributeTok{labels =}\NormalTok{ scales}\SpecialCharTok{::}\FunctionTok{percent}\NormalTok{(total\_percent))}

\NormalTok{minutes\_worn\_moduse }\OtherTok{\textless{}{-}}\NormalTok{ minutes\_worn}\SpecialCharTok{\%\textgreater{}\%}
  \FunctionTok{filter}\NormalTok{(usage }\SpecialCharTok{==} \StringTok{"moderate use"}\NormalTok{) }\SpecialCharTok{\%\textgreater{}\%}
  \FunctionTok{group\_by}\NormalTok{(worn) }\SpecialCharTok{\%\textgreater{}\%}
  \FunctionTok{summarise}\NormalTok{(}\AttributeTok{total =} \FunctionTok{n}\NormalTok{()) }\SpecialCharTok{\%\textgreater{}\%}
  \FunctionTok{mutate}\NormalTok{(}\AttributeTok{totals =} \FunctionTok{sum}\NormalTok{(total)) }\SpecialCharTok{\%\textgreater{}\%}
  \FunctionTok{group\_by}\NormalTok{(worn) }\SpecialCharTok{\%\textgreater{}\%}
  \FunctionTok{summarise}\NormalTok{(}\AttributeTok{total\_percent =}\NormalTok{ total }\SpecialCharTok{/}\NormalTok{ totals) }\SpecialCharTok{\%\textgreater{}\%}
  \FunctionTok{mutate}\NormalTok{(}\AttributeTok{labels =}\NormalTok{ scales}\SpecialCharTok{::}\FunctionTok{percent}\NormalTok{(total\_percent))}

\NormalTok{minutes\_worn\_lowuse }\OtherTok{\textless{}{-}}\NormalTok{ minutes\_worn}\SpecialCharTok{\%\textgreater{}\%}
  \FunctionTok{filter}\NormalTok{ (usage }\SpecialCharTok{==} \StringTok{"low use"}\NormalTok{) }\SpecialCharTok{\%\textgreater{}\%}
  \FunctionTok{group\_by}\NormalTok{(worn) }\SpecialCharTok{\%\textgreater{}\%}
  \FunctionTok{summarise}\NormalTok{(}\AttributeTok{total =} \FunctionTok{n}\NormalTok{()) }\SpecialCharTok{\%\textgreater{}\%}
  \FunctionTok{mutate}\NormalTok{(}\AttributeTok{totals =} \FunctionTok{sum}\NormalTok{(total)) }\SpecialCharTok{\%\textgreater{}\%}
  \FunctionTok{group\_by}\NormalTok{(worn) }\SpecialCharTok{\%\textgreater{}\%}
  \FunctionTok{summarise}\NormalTok{(}\AttributeTok{total\_percent =}\NormalTok{ total }\SpecialCharTok{/}\NormalTok{ totals) }\SpecialCharTok{\%\textgreater{}\%}
  \FunctionTok{mutate}\NormalTok{(}\AttributeTok{labels =}\NormalTok{ scales}\SpecialCharTok{::}\FunctionTok{percent}\NormalTok{(total\_percent))}

\NormalTok{minutes\_worn\_highuse}\SpecialCharTok{$}\NormalTok{worn }\OtherTok{\textless{}{-}} \FunctionTok{factor}\NormalTok{(minutes\_worn\_highuse}\SpecialCharTok{$}\NormalTok{worn, }\AttributeTok{levels =} \FunctionTok{c}\NormalTok{(}\StringTok{"All day"}\NormalTok{, }\StringTok{"More than half day"}\NormalTok{, }\StringTok{"Less than half day"}\NormalTok{))}
\NormalTok{minutes\_worn\_percent}\SpecialCharTok{$}\NormalTok{worn }\OtherTok{\textless{}{-}} \FunctionTok{factor}\NormalTok{(minutes\_worn\_percent}\SpecialCharTok{$}\NormalTok{worn, }\AttributeTok{levels =} \FunctionTok{c}\NormalTok{(}\StringTok{"All day"}\NormalTok{, }\StringTok{"More than half day"}\NormalTok{, }\StringTok{"Less than half day"}\NormalTok{))}
\NormalTok{minutes\_worn\_moduse}\SpecialCharTok{$}\NormalTok{worn }\OtherTok{\textless{}{-}} \FunctionTok{factor}\NormalTok{(minutes\_worn\_moduse}\SpecialCharTok{$}\NormalTok{worn, }\AttributeTok{levels =} \FunctionTok{c}\NormalTok{(}\StringTok{"All day"}\NormalTok{, }\StringTok{"More than half day"}\NormalTok{, }\StringTok{"Less than half day"}\NormalTok{))}
\NormalTok{minutes\_worn\_lowuse}\SpecialCharTok{$}\NormalTok{worn }\OtherTok{\textless{}{-}} \FunctionTok{factor}\NormalTok{(minutes\_worn\_lowuse}\SpecialCharTok{$}\NormalTok{worn, }\AttributeTok{levels =} \FunctionTok{c}\NormalTok{(}\StringTok{"All day"}\NormalTok{, }\StringTok{"More than half day"}\NormalTok{, }\StringTok{"Less than half day"}\NormalTok{))}

\FunctionTok{head}\NormalTok{(minutes\_worn\_highuse)}
\end{Highlighting}
\end{Shaded}

\begin{verbatim}
## # A tibble: 3 x 3
##   worn               total_percent labels
##   <fct>                      <dbl> <chr> 
## 1 All day                   0.498  49.8% 
## 2 Less than half day        0.0273 2.7%  
## 3 More than half day        0.474  47.4%
\end{verbatim}

\begin{Shaded}
\begin{Highlighting}[]
\FunctionTok{head}\NormalTok{(minutes\_worn\_moduse)}
\end{Highlighting}
\end{Shaded}

\begin{verbatim}
## # A tibble: 3 x 3
##   worn               total_percent labels
##   <fct>                      <dbl> <chr> 
## 1 All day                   0.649  65%   
## 2 Less than half day        0.0175 2%    
## 3 More than half day        0.333  33%
\end{verbatim}

\begin{Shaded}
\begin{Highlighting}[]
\FunctionTok{head}\NormalTok{(minutes\_worn\_lowuse)}
\end{Highlighting}
\end{Shaded}

\begin{verbatim}
## # A tibble: 2 x 3
##   worn               total_percent labels
##   <fct>                      <dbl> <chr> 
## 1 All day                     0.75 75%   
## 2 More than half day          0.25 25%
\end{verbatim}

Now that we have created the four data frames and also ordered worn
level categories, we can visualize our results in the following plots.
All the plots have been arranged together for a better visualization.

\begin{Shaded}
\begin{Highlighting}[]
\FunctionTok{ggarrange}\NormalTok{(}
  \FunctionTok{ggplot}\NormalTok{(minutes\_worn\_percent, }\FunctionTok{aes}\NormalTok{(}\AttributeTok{x=}\StringTok{""}\NormalTok{,}\AttributeTok{y=}\NormalTok{total\_percent, }\AttributeTok{fill=}\NormalTok{worn)) }\SpecialCharTok{+}
  \FunctionTok{geom\_bar}\NormalTok{(}\AttributeTok{stat =} \StringTok{"identity"}\NormalTok{, }\AttributeTok{width =} \DecValTok{1}\NormalTok{)}\SpecialCharTok{+}
  \FunctionTok{coord\_polar}\NormalTok{(}\StringTok{"y"}\NormalTok{, }\AttributeTok{start=}\DecValTok{0}\NormalTok{)}\SpecialCharTok{+}
  \FunctionTok{theme\_minimal}\NormalTok{()}\SpecialCharTok{+}
  \FunctionTok{theme}\NormalTok{(}\AttributeTok{axis.title.x=} \FunctionTok{element\_blank}\NormalTok{(),}
        \AttributeTok{axis.title.y =} \FunctionTok{element\_blank}\NormalTok{(),}
        \AttributeTok{panel.border =} \FunctionTok{element\_blank}\NormalTok{(), }
        \AttributeTok{panel.grid =} \FunctionTok{element\_blank}\NormalTok{(), }
        \AttributeTok{axis.ticks =} \FunctionTok{element\_blank}\NormalTok{(),}
        \AttributeTok{axis.text.x =} \FunctionTok{element\_blank}\NormalTok{(),}
        \AttributeTok{plot.title =} \FunctionTok{element\_text}\NormalTok{(}\AttributeTok{hjust =} \FloatTok{0.5}\NormalTok{, }\AttributeTok{size=}\DecValTok{14}\NormalTok{, }\AttributeTok{face =} \StringTok{"bold"}\NormalTok{),}
        \AttributeTok{plot.subtitle =} \FunctionTok{element\_text}\NormalTok{(}\AttributeTok{hjust =} \FloatTok{0.5}\NormalTok{)) }\SpecialCharTok{+}
    \FunctionTok{scale\_fill\_manual}\NormalTok{(}\AttributeTok{values =} \FunctionTok{c}\NormalTok{(}\StringTok{"\#004d99"}\NormalTok{, }\StringTok{"\#3399ff"}\NormalTok{, }\StringTok{"\#cce6ff"}\NormalTok{))}\SpecialCharTok{+}
  \FunctionTok{geom\_text}\NormalTok{(}\FunctionTok{aes}\NormalTok{(}\AttributeTok{label =}\NormalTok{ labels),}
            \AttributeTok{position =} \FunctionTok{position\_stack}\NormalTok{(}\AttributeTok{vjust =} \FloatTok{0.5}\NormalTok{), }\AttributeTok{size =} \FloatTok{3.5}\NormalTok{)}\SpecialCharTok{+}
  \FunctionTok{labs}\NormalTok{(}\AttributeTok{title=}\StringTok{"Time worn per day"}\NormalTok{, }\AttributeTok{subtitle =} \StringTok{"Total Users"}\NormalTok{),}
  \FunctionTok{ggarrange}\NormalTok{(}
  \FunctionTok{ggplot}\NormalTok{(minutes\_worn\_highuse, }\FunctionTok{aes}\NormalTok{(}\AttributeTok{x=}\StringTok{""}\NormalTok{,}\AttributeTok{y=}\NormalTok{total\_percent, }\AttributeTok{fill=}\NormalTok{worn)) }\SpecialCharTok{+}
  \FunctionTok{geom\_bar}\NormalTok{(}\AttributeTok{stat =} \StringTok{"identity"}\NormalTok{, }\AttributeTok{width =} \DecValTok{1}\NormalTok{)}\SpecialCharTok{+}
  \FunctionTok{coord\_polar}\NormalTok{(}\StringTok{"y"}\NormalTok{, }\AttributeTok{start=}\DecValTok{0}\NormalTok{)}\SpecialCharTok{+}
  \FunctionTok{theme\_minimal}\NormalTok{()}\SpecialCharTok{+}
  \FunctionTok{theme}\NormalTok{(}\AttributeTok{axis.title.x=} \FunctionTok{element\_blank}\NormalTok{(),}
        \AttributeTok{axis.title.y =} \FunctionTok{element\_blank}\NormalTok{(),}
        \AttributeTok{panel.border =} \FunctionTok{element\_blank}\NormalTok{(), }
        \AttributeTok{panel.grid =} \FunctionTok{element\_blank}\NormalTok{(), }
        \AttributeTok{axis.ticks =} \FunctionTok{element\_blank}\NormalTok{(),}
        \AttributeTok{axis.text.x =} \FunctionTok{element\_blank}\NormalTok{(),}
        \AttributeTok{plot.title =} \FunctionTok{element\_text}\NormalTok{(}\AttributeTok{hjust =} \FloatTok{0.5}\NormalTok{, }\AttributeTok{size=}\DecValTok{14}\NormalTok{, }\AttributeTok{face =} \StringTok{"bold"}\NormalTok{),}
        \AttributeTok{plot.subtitle =} \FunctionTok{element\_text}\NormalTok{(}\AttributeTok{hjust =} \FloatTok{0.5}\NormalTok{), }
        \AttributeTok{legend.position =} \StringTok{"none"}\NormalTok{)}\SpecialCharTok{+}
    \FunctionTok{scale\_fill\_manual}\NormalTok{(}\AttributeTok{values =} \FunctionTok{c}\NormalTok{(}\StringTok{"\#004d99"}\NormalTok{, }\StringTok{"\#3399ff"}\NormalTok{, }\StringTok{"\#cce6ff"}\NormalTok{))}\SpecialCharTok{+}
  \FunctionTok{geom\_text\_repel}\NormalTok{(}\FunctionTok{aes}\NormalTok{(}\AttributeTok{label =}\NormalTok{ labels),}
            \AttributeTok{position =} \FunctionTok{position\_stack}\NormalTok{(}\AttributeTok{vjust =} \FloatTok{0.5}\NormalTok{), }\AttributeTok{size =} \DecValTok{3}\NormalTok{)}\SpecialCharTok{+}
  \FunctionTok{labs}\NormalTok{(}\AttributeTok{title=}\StringTok{""}\NormalTok{, }\AttributeTok{subtitle =} \StringTok{"High use {-} Users"}\NormalTok{), }
  \FunctionTok{ggplot}\NormalTok{(minutes\_worn\_moduse, }\FunctionTok{aes}\NormalTok{(}\AttributeTok{x=}\StringTok{""}\NormalTok{,}\AttributeTok{y=}\NormalTok{total\_percent, }\AttributeTok{fill=}\NormalTok{worn)) }\SpecialCharTok{+}
  \FunctionTok{geom\_bar}\NormalTok{(}\AttributeTok{stat =} \StringTok{"identity"}\NormalTok{, }\AttributeTok{width =} \DecValTok{1}\NormalTok{)}\SpecialCharTok{+}
  \FunctionTok{coord\_polar}\NormalTok{(}\StringTok{"y"}\NormalTok{, }\AttributeTok{start=}\DecValTok{0}\NormalTok{)}\SpecialCharTok{+}
  \FunctionTok{theme\_minimal}\NormalTok{()}\SpecialCharTok{+}
  \FunctionTok{theme}\NormalTok{(}\AttributeTok{axis.title.x=} \FunctionTok{element\_blank}\NormalTok{(),}
        \AttributeTok{axis.title.y =} \FunctionTok{element\_blank}\NormalTok{(),}
        \AttributeTok{panel.border =} \FunctionTok{element\_blank}\NormalTok{(), }
        \AttributeTok{panel.grid =} \FunctionTok{element\_blank}\NormalTok{(), }
        \AttributeTok{axis.ticks =} \FunctionTok{element\_blank}\NormalTok{(),}
        \AttributeTok{axis.text.x =} \FunctionTok{element\_blank}\NormalTok{(),}
        \AttributeTok{plot.title =} \FunctionTok{element\_text}\NormalTok{(}\AttributeTok{hjust =} \FloatTok{0.5}\NormalTok{, }\AttributeTok{size=}\DecValTok{14}\NormalTok{, }\AttributeTok{face =} \StringTok{"bold"}\NormalTok{), }
        \AttributeTok{plot.subtitle =} \FunctionTok{element\_text}\NormalTok{(}\AttributeTok{hjust =} \FloatTok{0.5}\NormalTok{),}
        \AttributeTok{legend.position =} \StringTok{"none"}\NormalTok{) }\SpecialCharTok{+}
    \FunctionTok{scale\_fill\_manual}\NormalTok{(}\AttributeTok{values =} \FunctionTok{c}\NormalTok{(}\StringTok{"\#004d99"}\NormalTok{, }\StringTok{"\#3399ff"}\NormalTok{, }\StringTok{"\#cce6ff"}\NormalTok{))}\SpecialCharTok{+}
  \FunctionTok{geom\_text}\NormalTok{(}\FunctionTok{aes}\NormalTok{(}\AttributeTok{label =}\NormalTok{ labels),}
            \AttributeTok{position =} \FunctionTok{position\_stack}\NormalTok{(}\AttributeTok{vjust =} \FloatTok{0.5}\NormalTok{), }\AttributeTok{size =} \DecValTok{3}\NormalTok{)}\SpecialCharTok{+}
  \FunctionTok{labs}\NormalTok{(}\AttributeTok{title=}\StringTok{""}\NormalTok{, }\AttributeTok{subtitle =} \StringTok{"Moderate use {-} Users"}\NormalTok{), }
  \FunctionTok{ggplot}\NormalTok{(minutes\_worn\_lowuse, }\FunctionTok{aes}\NormalTok{(}\AttributeTok{x=}\StringTok{""}\NormalTok{,}\AttributeTok{y=}\NormalTok{total\_percent, }\AttributeTok{fill=}\NormalTok{worn)) }\SpecialCharTok{+}
  \FunctionTok{geom\_bar}\NormalTok{(}\AttributeTok{stat =} \StringTok{"identity"}\NormalTok{, }\AttributeTok{width =} \DecValTok{1}\NormalTok{)}\SpecialCharTok{+}
  \FunctionTok{coord\_polar}\NormalTok{(}\StringTok{"y"}\NormalTok{, }\AttributeTok{start=}\DecValTok{0}\NormalTok{)}\SpecialCharTok{+}
  \FunctionTok{theme\_minimal}\NormalTok{()}\SpecialCharTok{+}
  \FunctionTok{theme}\NormalTok{(}\AttributeTok{axis.title.x=} \FunctionTok{element\_blank}\NormalTok{(),}
        \AttributeTok{axis.title.y =} \FunctionTok{element\_blank}\NormalTok{(),}
        \AttributeTok{panel.border =} \FunctionTok{element\_blank}\NormalTok{(), }
        \AttributeTok{panel.grid =} \FunctionTok{element\_blank}\NormalTok{(), }
        \AttributeTok{axis.ticks =} \FunctionTok{element\_blank}\NormalTok{(),}
        \AttributeTok{axis.text.x =} \FunctionTok{element\_blank}\NormalTok{(),}
        \AttributeTok{plot.title =} \FunctionTok{element\_text}\NormalTok{(}\AttributeTok{hjust =} \FloatTok{0.5}\NormalTok{, }\AttributeTok{size=}\DecValTok{14}\NormalTok{, }\AttributeTok{face =} \StringTok{"bold"}\NormalTok{), }
        \AttributeTok{plot.subtitle =} \FunctionTok{element\_text}\NormalTok{(}\AttributeTok{hjust =} \FloatTok{0.5}\NormalTok{),}
        \AttributeTok{legend.position =} \StringTok{"none"}\NormalTok{) }\SpecialCharTok{+}
    \FunctionTok{scale\_fill\_manual}\NormalTok{(}\AttributeTok{values =} \FunctionTok{c}\NormalTok{(}\StringTok{"\#004d99"}\NormalTok{, }\StringTok{"\#3399ff"}\NormalTok{, }\StringTok{"\#cce6ff"}\NormalTok{))}\SpecialCharTok{+}
  \FunctionTok{geom\_text}\NormalTok{(}\FunctionTok{aes}\NormalTok{(}\AttributeTok{label =}\NormalTok{ labels),}
            \AttributeTok{position =} \FunctionTok{position\_stack}\NormalTok{(}\AttributeTok{vjust =} \FloatTok{0.5}\NormalTok{), }\AttributeTok{size =} \DecValTok{3}\NormalTok{)}\SpecialCharTok{+}
  \FunctionTok{labs}\NormalTok{(}\AttributeTok{title=}\StringTok{""}\NormalTok{, }\AttributeTok{subtitle =} \StringTok{"Low use {-} Users"}\NormalTok{), }
  \AttributeTok{ncol =} \DecValTok{3}\NormalTok{),}
  \AttributeTok{nrow =} \DecValTok{2}\NormalTok{)}
\end{Highlighting}
\end{Shaded}

\includegraphics{bellabeat_notebook_files/figure-latex/unnamed-chunk-28-1.pdf}

Per our plots we can see that 51\% of the total of users wear the device
all day long, 46\% more than half day long and just 3\% less than half
day.

Just a reminder

\begin{itemize}
\tightlist
\item
  high use - users who use their device between 21 and 31 days.
\item
  moderate use - users who use their device between 10 and 20 days.
\item
  low use - users who use their device between 1 and 10 days.
\end{itemize}

If we filter the total users considering the days they have used the
device and also check each day how long they have worn the device, we
have the following results:

\begin{itemize}
\tightlist
\item
  High users: About 50\% of the users that have used their device
  between 21 and 31 days wear it all day. 47.4\% use the device more
  than half day but not all day.
\item
  Moderate and low users wear the device more on the days they use it.
\end{itemize}

\hypertarget{market-competition}{%
\subsubsection{MARKET COMPETITION}\label{market-competition}}

We use the excel sheet which has data gathered by research on other
products similar to Bellabeat's SPRING.

\begin{Shaded}
\begin{Highlighting}[]
\FunctionTok{ggplot}\NormalTok{(products, }\FunctionTok{aes}\NormalTok{(}\AttributeTok{x =}\NormalTok{ ratings, }\AttributeTok{y =}\NormalTok{ price,}\AttributeTok{color =}\NormalTok{ product)) }\SpecialCharTok{+}
  \FunctionTok{geom\_point}\NormalTok{() }\SpecialCharTok{+}
  \FunctionTok{theme}\NormalTok{(}\AttributeTok{plot.title =} \FunctionTok{element\_text}\NormalTok{(}\AttributeTok{hjust =} \FloatTok{0.5}\NormalTok{, }\AttributeTok{size=}\DecValTok{14}\NormalTok{, }\AttributeTok{face =} \StringTok{"bold"}\NormalTok{)) }\SpecialCharTok{+}
  \FunctionTok{labs}\NormalTok{(}\AttributeTok{title =} \StringTok{"Market Competitors"}\NormalTok{)}
\end{Highlighting}
\end{Shaded}

\includegraphics{bellabeat_notebook_files/figure-latex/unnamed-chunk-29-1.pdf}

\hypertarget{phase6act}{%
\subsection{PHASE6:ACT}\label{phase6act}}

Bellabeat's mission is deeply rooted in empowering women through
data-driven insights. To effectively support Bellabeat's mission and
address our business objectives, it is imperative to harness our own
comprehensive tracking data for in-depth analysis. The data sets we've
employed thus far have limitations, primarily their small sample size
and the absence of user demographic information. As our primary target
demographic comprises young and adult women, it is crucial to persist in
uncovering actionable trends within our data sets. This ongoing pursuit
of insights will enable us to craft a focused and effective marketing
strategy that resonates with our core audience, ensuring that we
continue to serve and empower women in their health and wellness
journeys.

That being said, after our analysis we have found different trends that
may help our online campaign and improve \emph{Bellabeat SPRING}

\hypertarget{features-to-be-improved}{%
\subsubsection{FEATURES TO BE IMPROVED}\label{features-to-be-improved}}

On our analysis we didn't just check trends on daily users habits we
also realized that 88\% of the users use their device on a daily basis
and that 50\% of the users wear the device all time the day they used
it. We can continue promote Bellabeat's products features:

\begin{itemize}
\tightlist
\item
  Water-resistant: Enhancing the water-resistant capabilities of the
  product will ensure it remains functional and reliable, even in wet or
  humid conditions.
\item
  Long-lasting batteries: Improving battery life is essential for
  providing users with a more reliable and longer-lasting experience,
  reducing the need for frequent recharging.
\item
  Fashion/elegant product: Elevating the design and aesthetics of the
  product will make it more appealing, combining style with
  functionality for a more attractive user experience.
\item
  Insulation: Enhancing insulation features can better protect the
  product's internal components from temperature variations and
  environmental factors, ensuring consistent performance.
\item
  Bellabeat app auto-sync feature: Implementing an automatic syncing
  feature in the Bellabeat app will streamline data transfer and make it
  more convenient for users to access their health and wellness
  information.
\end{itemize}

\hypertarget{recommendations}{%
\subsubsection{RECOMMENDATIONS}\label{recommendations}}

\begin{itemize}
\tightlist
\item
  Unique feature:Incorporating these innovative features can enhance the
  functionalityand desirability of smart water bottles, making them more
  attractive to consumers who want to maintain a healthy and eco
  conscious lifestyle

  \begin{itemize}
  \tightlist
  \item
    Emergency Features: Include a distress signal or emergency beacon
    for outdoor enthusiasts or those in need of assistance.
  \item
    Leak Detection: Incorporate sensors to detect leaks and send alerts
    to prevent messes and water wastage.
  \item
    Social Sharing: Enable users to share their hydration achievements
    and challenges with friends and social networks.
  \item
    Eco-Friendly Materials: Use sustainable and recyclable materials in
    the bottle's construction to reduce its environmental impact.
  \end{itemize}
\item
  Challenges between users:To promote usage of the logged activities
  feature, Bellabeat could advertise challenges between users to promote
  usage.If the budget permits, there could be an incentive involved with
  winning a certain amount of challenges.
\end{itemize}

\end{document}
